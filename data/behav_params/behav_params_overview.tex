% Options for packages loaded elsewhere
\PassOptionsToPackage{unicode}{hyperref}
\PassOptionsToPackage{hyphens}{url}
\documentclass[
]{article}
\usepackage{xcolor}
\usepackage[margin=1in]{geometry}
\usepackage{amsmath,amssymb}
\setcounter{secnumdepth}{-\maxdimen} % remove section numbering
\usepackage{iftex}
\ifPDFTeX
  \usepackage[T1]{fontenc}
  \usepackage[utf8]{inputenc}
  \usepackage{textcomp} % provide euro and other symbols
\else % if luatex or xetex
  \usepackage{unicode-math} % this also loads fontspec
  \defaultfontfeatures{Scale=MatchLowercase}
  \defaultfontfeatures[\rmfamily]{Ligatures=TeX,Scale=1}
\fi
\usepackage{lmodern}
\ifPDFTeX\else
  % xetex/luatex font selection
\fi
% Use upquote if available, for straight quotes in verbatim environments
\IfFileExists{upquote.sty}{\usepackage{upquote}}{}
\IfFileExists{microtype.sty}{% use microtype if available
  \usepackage[]{microtype}
  \UseMicrotypeSet[protrusion]{basicmath} % disable protrusion for tt fonts
}{}
\makeatletter
\@ifundefined{KOMAClassName}{% if non-KOMA class
  \IfFileExists{parskip.sty}{%
    \usepackage{parskip}
  }{% else
    \setlength{\parindent}{0pt}
    \setlength{\parskip}{6pt plus 2pt minus 1pt}}
}{% if KOMA class
  \KOMAoptions{parskip=half}}
\makeatother
\usepackage{graphicx}
\makeatletter
\newsavebox\pandoc@box
\newcommand*\pandocbounded[1]{% scales image to fit in text height/width
  \sbox\pandoc@box{#1}%
  \Gscale@div\@tempa{\textheight}{\dimexpr\ht\pandoc@box+\dp\pandoc@box\relax}%
  \Gscale@div\@tempb{\linewidth}{\wd\pandoc@box}%
  \ifdim\@tempb\p@<\@tempa\p@\let\@tempa\@tempb\fi% select the smaller of both
  \ifdim\@tempa\p@<\p@\scalebox{\@tempa}{\usebox\pandoc@box}%
  \else\usebox{\pandoc@box}%
  \fi%
}
% Set default figure placement to htbp
\def\fps@figure{htbp}
\makeatother
\setlength{\emergencystretch}{3em} % prevent overfull lines
\providecommand{\tightlist}{%
  \setlength{\itemsep}{0pt}\setlength{\parskip}{0pt}}
\usepackage{placeins}
\usepackage{booktabs}
\usepackage{longtable}
\usepackage{array}
\usepackage{multirow}
\usepackage{wrapfig}
\usepackage{float}
\usepackage{colortbl}
\usepackage{pdflscape}
\usepackage{tabu}
\usepackage{threeparttable}
\usepackage{threeparttablex}
\usepackage[normalem]{ulem}
\usepackage{makecell}
\usepackage{xcolor}
\usepackage{siunitx}

    \newcolumntype{d}{S[
      table-align-text-before=false,
      table-align-text-after=false,
      input-symbols={-,\*+()}
    ]}
  
\usepackage{bookmark}
\IfFileExists{xurl.sty}{\usepackage{xurl}}{} % add URL line breaks if available
\urlstyle{same}
\hypersetup{
  hidelinks,
  pdfcreator={LaTeX via pandoc}}

\author{}
\date{\vspace{-2.5em}}

\begin{document}

The model's central functionality relies on a variety of behavioral
mechanisms that are parametrised econometrically using micro data.
Furthermore, the model relies on additional sources of micro data for
validation. In the following section we outline, first, the data and
analysis used to derive the model's behavioral parameters, followed by
an inventory of the data used to validate various model outputs. We note
where we make use of methodology employed or constructed by other
authors.

\subsection{Application Effort and Learning Dynamics: Applications
Sent}\label{application-effort-and-learning-dynamics-applications-sent}

First, we employ data from the US Bureau of Labor Statistics and the
Curren Population Survey on the application effort of unemployed
job-seekers to discipline our behavioral mechanism for search. More
specifically, in 2018 (May, September) and 2020 (February, May), the
Bureau of Labor Statistics ran the ``Unemployment Insurance Nonfilers''
supplemental survey to the monthly Current Population Survey run by the
US Census Bureau. The survey's stated intent was to ``obtain information
on the characteristics of people who do not file for Unemployment
Insurance benefits as well as their reasons for not doing so.'' The
survey was conducted for all persons responding the the monthly Current
Population Survey which encompasses ``all persons in the civilian
non-institutional population of the United States living in households.
The probability sample selected to represent the universe consists of
approximately 54,000 households.''

Relevant to our work, survey respondents were asked the following two
questions:

\begin{figure}
\centering
\pandocbounded{\includegraphics[keepaspectratio]{CPS_BLS_Supplement_18_22/prunedur_q.png}}
\caption{Survey Question: Unemployment Duration}
\end{figure}

\begin{figure}
\centering
\pandocbounded{\includegraphics[keepaspectratio]{CPS_BLS_Supplement_18_22/applications_sent_q.png}}
\caption{Survey Question: Applications Sent}
\end{figure}

\subsubsection{Overview of Survey
Results}\label{overview-of-survey-results}

First, we replicate figures produced for a 2020 Bureau of Labor
Statistics ``Beyond the Numbers'' issue 2020
\href{https://www.bls.gov/opub/btn/volume-9/how-do-jobseekers-search-for-jobs.htm\#_edn5}{``Beyond
the Numbers'' issue}, illustrating some high-level results from the
survey, prior to describing the econometric specifications employed
using the raw survey data. As seen in the image above, the survey
responses regarding applications sent are ``binned'' into intervals (ie.
number of people sending 81 or more applications or unemployment
duration of between 5 and 14 weeks) which means that any line plots (or
linear interpretation of the bar graph) should be done with caution.

In Figure 1, the top left panel shows the proportion of all individuals
sending X amount of applications receiving Y amount of interviews. The
plot indicates a ``consistent'' return to sending more applications,
although as demonstrated in the bottom left plot, the number of
applications sent is not a linear predictor of job offers received. More
precisely, the bottom left plot demonstrates that the percentage of
jobseekers receiving an offer seems to increase as a function of the
number of applications sent, until a certain point.Next, the right plot
demonstrates the number of applications sent (red), interviews received
(green), average interview:application ratio (blue), and probability of
receiving a job offer (purple) by individuals in each category of
unemployment duration. There is some indication that both effort and
success seem to increase and then decline with time spent in
unemployment, apart from success as measured by receiving a job offer
which seems to consistently decline with time spent in unemployment.

\begin{figure}
\centering
\pandocbounded{\includegraphics[keepaspectratio]{behav_params_overview_files/figure-latex/cpssupp1-1.png}}
\caption{Replication of BLS Analysis}
\end{figure}

\subsubsection{Econometric Specification Using Raw
Data}\label{econometric-specification-using-raw-data}

\pandocbounded{\includegraphics[keepaspectratio]{behav_params_overview_files/figure-latex/cps_supp2-1.png}}

To inform our agent behavior, we derive an unemployment-duration
dependent measure of application effort, reporting in applications sent.
More precisely, we estimate the probability distribution over reported
job application intensity during unemployment using pooled micro data
from the 2018 and 2022 waves of the CPS in which the Bureau of Labor
Statistics conducted a Job Search Supplement. The survey asks unemployed
respondents who are actively searching for work the amount of job
applications they have sent. Respondents report job application counts
in ordinal bins:
\texttt{0\textquotesingle{}\textquotesingle{},}1--10'\,`,
\texttt{11–20\textquotesingle{}\textquotesingle{},}21--80'`, and ``81 or
more'\,'. To account for the lack of a continuous dependent variable, we
estimate a series of ordinal logistic regression models to recover the
conditional probability of each response bin as a function of
unemployment duration and various demographic characteristics. We test
model specifications along three dimensions: (i) link function,
comparing logistic, probit, complementary log-log (cloglog), and log-log
links; (ii) linear, quadratic, and cubic specifications of unemployment
duration; and (iii) models with and without demographic covariates
(education, gender, age, and family income; race was excluded due to
lack of statistical significance across models). Formally, the model
estimates \(\Pr(Y_i \leq j \mid X_i)\), the cumulative probability of
observing response category \(Y_i\) for individual \(i\) below \(j\)
where \(j\) represents the five ordinal bins given various
transformations of the vector \(X_i\) of independent variables
(unemployment duration and demographic controls).

Below, we display the results of an exploration of the probability of
reporting a specific number of applications sent (in the bins as in the
survey question above) using various specifications of an ordinal
logistic regression. Wetest specifications varying three different model
parameters:\\
1. link function\\
2. linear vs.~quadratic unemploymentduration,\\
3. with and without demographic control variables (education, gender,
age, family income - race excluded because of lack of statistical
significance though this can be revisited.)

We estimate an ordinal logistic regression model for reported
applications sent \(Y_i\) in \({0, 1, 2, 3, 4}\) testing four different
link functions: the complementary log-log (cloglog), logistic, log-log,
and probit link functions. Let \(X_i^\top \beta\) denote the predictor
variable. The cumulative probability of observing response category
\(j\) or below, \(\Pr(Y_i \leq j \mid X_i)\), is modeled as follows for
each link function:

\begin{align*}
\text{Complementary log-log (cloglog):} \quad & \Pr(Y_i \leq j \mid X_i) = 1 - \exp\left( -\exp\left( \tau_j - X_i^\top \beta \right) \right) \\
\text{Logistic (logit):} \quad & \Pr(Y_i \leq j \mid X_i) = \frac{1}{1 + \exp\left( -(\tau_j - X_i^\top \beta) \right)} \\
\text{Loglog:} \quad & \Pr(Y_i \leq j \mid X_i) = \exp\left( -\exp\left( -(\tau_j - X_i^\top \beta) \right) \right) \\
\text{Probit:} \quad & \Pr(Y_i \leq j \mid X_i) = \Phi(\tau_j - X_i^\top \beta)
\end{align*}

Here, \(\Phi(\cdot)\) denotes the cumulative distribution function of
the standard normal distribution. The estimated coefficients \(\beta\)
are interpreted conditional on the choice of link function where \(X_i\)
is either:

\(X_i = \left( \text{Unemp.Dur.}_i \right)\)

\(X_i = \left( \text{Unemp.Dur.}_i^2 \right)\)

\(X_i = \left( \text{Unemp.Dur.}_i, \text{Unemp.Dur.}_i^2 \right)\)

with and without control variables (education, gender, age, family
income).

Assumptions about the probability distribution of the errors associated
with each link function:\\
- \emph{Logit:} Useful when responses are evenly distributed across
categories.\\
- \emph{Probit:} Useful when latent variable is assumed to be normally
distributed.\\
- \emph{Complementary log-log:} Useful when higher categories are more
probable.\\
- \emph{Log-log:} When early categories are of more importance or more
probable.

\pandocbounded{\includegraphics[keepaspectratio]{behav_params_overview_files/figure-latex/cps_supp3-1.png}}
\pandocbounded{\includegraphics[keepaspectratio]{behav_params_overview_files/figure-latex/cps_supp3-2.png}}
\pandocbounded{\includegraphics[keepaspectratio]{behav_params_overview_files/figure-latex/cps_supp3-3.png}}

Using an AIC information criterion to compare the fit across all models,
clear results emerge. Models including socio-demographic controls
consistently outperform unadjusted models (blue versus red dots in the
figure below) and the inclusion of a quadratic transformation (labelled
``Lin-Quadratic'' in the plot below) of unemployment duration better
captures the non-linear relationship between unemployment duration and
application effort. Among link functions, the complementary log-log
specification performs best across model comparisons. Though the
logistic link function emerges as slightly superior in the specification
incorporating a linear and quadratic term, we choose to employ the
complementary log-log link function to align with the hypothesis that
fine-grained resolution is needed among low application effort
categories, which dominate the data. Thus, employing a complementary
log-log link function, quadratic unemployment duration, and full
demographic controls, we generate predicted probabilities over the five
application bins for unemployment spells ranging from 0 to 36 months.
These fitted probabilities serve as the empirical foundation for
modeling job search effort in the agent-based simulation. In our chosen
specification, the odds of reporting a lower application bin increase by
approximately 0.1\% per additional month unemployed, a relationship
statistically significant at the 0.1\% level. However, the inclusion of
a quadratic term allows for a concave shape to emerge, better fitting
the non-linearity of this relationship between unemployment duration and
applications sent.

\pandocbounded{\includegraphics[keepaspectratio]{behav_params_overview_files/figure-latex/cps_supp4-1.png}}

The figure below demonstrates the predicted probability distribution of
application effort by unemployment duration indicating a non-linear
concave search effort. We believe this contributes to an open debate in
the job search literature regarding the shape of search effort over the
unemployment spell. The concave application effort emerging from this
data aligns with previous observations about unemployed workers engaging
in delayed search while either grieving job loss or engaging in job
search planning and adjusting expectations about their re-employment
prospects, as described in the main text.

The final result is that for each additional quarter of unemployment, an
individual's odds of dropping to a lower-level application category
decreases by \textasciitilde.1\%. This is statistically significant
across all specifications at the 0.1\% level.

\pandocbounded{\includegraphics[keepaspectratio]{behav_params_overview_files/figure-latex/unnamed-chunk-1-1.png}}

\subsection{Wage Expectations and Satisficing: Reservation Wage
Adjustment}\label{wage-expectations-and-satisficing-reservation-wage-adjustment}

As part of the Current Population Survey, the US Census Bureau conducts
an annual Displaced Worker Supplement in which workers who have lost
their job in the last three years are asked additional questions about
their unemployment experiences and (if re-employed) their re-employment
conditions. From this we draw a reservation wage adjustment rate as a
function of unemployment duration. We compare various econometric
specifications across several samples that correct for selection effects
that typically confound studies of duration-dependent employment
outcomes.

As part of the Current Population Survey, the US Census Bureau conducts
an annual
\href{https://cps.ipums.org/cps/dw_sample_notes.shtml}{Displaced Worker
Supplement} in which workers who have lost their job in the last three
years are asked additional questions about their unemployment
experiences and (if re-employed) their re-employment conditions.

As reported in the survey documentation linked above, ``the universe for
the Displaced Workers Supplement is civilians 20 or older. Respondents
are further categorized as a `displaced worker' if they meet additional
characteristics (see DWSTAT). Users should note that there is an
important difference in definition of displaced worker across samples.
Before 1994, displaced workers are those who lost or left a job during
the past 5 years. After 1994, displaced workers are those who lost or
left a job due to layoffs or shutdowns within the past 3 years. For 1998
on, respondents are only considered displaced workers if they had lost
or left a job due to layoffs or shutdowns within the past 3 years, were
not self-employed, and did not expect to be recalled to work within the
next six months. Self-response was not required for this supplement
after 1994, so often one individual answered for all household
members.''

We utilize the information reported on an individual's weekly wage at
their lost job, wage at their new job, and the time spent unemployed to
derive a measure of duration-dependent reservation wage adjustment. More
precisely, we regress the ratio of the new wage to the wage at the lost
job on unemployment duration and various control variables in a
cross-sectional setting. We compare the model fit across linear,
quadratic, and cubic specifications, with and without various
combinations of control variables (whether or not an individual received
unemployment compensation, age, race, sex, marital status, education,
previous wage level). Note that wages are reported in hourly and weekly
values but this reporting is inconsistent across observations. In other
words, though most individuals (4600/6198) report their wage in both
units, 270 report only hourly and 1328 report only weekly. To be able to
combine information on all workers to one value, we select the present
statistic for those missing one and retain either the minimum, maximum,
or mean of the hourly versus weekly wage for those reporting both. We
display box plots of these wage ratios across unemployment duration bins
for the different methods of reconciling the missing data later in this
document. The data used below is from annual survey responses between
2000-2025. We use the supplement sample weights in all results below.

We note where the sample has been trimmed for outliers (wage ratio
between {[}0.25, 2{]} and unemployment duration less than 96 weeks
(\textasciitilde24 months). All analysis below uses Displaced Worker
Sample Weights to ensure appropriate weighting of survey responses and
reduce any influence of selection bias.

Below, we outline the data cleaning procedure, provide descriptive
figures and statistics, outline the econometric estimation strategy,
provide regression results using the raw sample and reweighted samples
addressing selection issues and non-uniformity, and provide information
on the representativeness of the raw sample. The sample is non-uniform
in unemployment duration (less observations are observed for higher
values of unemployment duration). We employ three methods of reweighting
to address these selection issues (Heckman Selection correction,
entropy-balancing, and propensity score matching) to deal with
representativeness issues of across values of unemployment durations.
These re-weighting and sample balancing methods confirm the
directionality of the regression results in the non-uniform sample,
providing greater confidence in the triangulated reservation wage
adjustment rate.

Overall, we find that individuals accept a \textasciitilde1-percentage
point decrease in the wage ratio per additional month of unemployment.
Variations using model reweighting, different samples, combinations of
control variables, reported hourly and weekly wage ratios do not seem to
affect the result. However, the data seems to follow a non-linear
relationship (we see little satisficing until around \textasciitilde12
months of unemployment) after which the wage ratio begins to decrease.
Individuals seem to accept a below-1 relative wage ratio (current
wage:wage at lost job) following a year of unemployment.

\textbf{Potential Limitations:}

\begin{enumerate}
\def\labelenumi{\arabic{enumi}.}
\tightlist
\item
  \textbf{Displaced worker classification as outlined above.} We do not
  distinguish between workers in our model that are voluntarily or
  involuntarily separated from their jobs. Therefore, the displaced
  worker classification outlined above does not represent individuals
  unemployed voluntarily.
\item
  \textbf{The reported `current wage' is not necessarily the realised
  wage post-re-employment.} Individuals report the wage at their lost
  job, the amount of time unemployed until they were re-employed, and
  the wage they hold at their current job. However, it is not indicated
  whether the current job is the same job as the first they were
  re-employed at. As such, there is uncertainty in the measurement of
  this outcome as an accepted wage that is relatively low compared to an
  individual's previous wage might be a temporary reality rather than a
  true re-employment wage (i.e., an individual finding stop-gap
  employment).
\item
  \textbf{Outcome variable:} The outcome variable does not adequately
  handle fundamentally different wage scales (i.e., a 10\% wage increase
  would likely be more or less devastating depending on the initial wage
  level). We control for wage levels in various specifications listed
  below. We find that controlling for wage levels does not significantly
  impact our results.
\end{enumerate}

\section{Descriptives}\label{descriptives}

First, we display the distribution of continuous (red) and binned (blue)
unemployment duration. The distribution is, as expected, heavily skewed,
with more individuals concentrated at low unemployment durations. The
binned values in the blue histogram are the binned values later employed
as an outcome variable in various regressions.

\begin{center}\includegraphics[width=0.9\linewidth]{behav_params_overview_files/figure-latex/disc_descriptives-1} \end{center}

Next, looking at the reported wage ratios in weekly and hourly values
(without reconciling the missing data), the mean is fixed near 1 until
\textgreater12 mos of unemployment in hourly wage reporting. In weekly
wage reporting, the ``satisficing'' seems to start earlier in
unemployment duration, indicating that the relationship is potentially
negative and non-linear.

\begin{center}\includegraphics[width=0.9\linewidth]{behav_params_overview_files/figure-latex/disc_descriptives2-1} \end{center}

Next, we compare the various options for reconciling missing data across
survey responses (i.e., when either weekly or hourly wage is reported
but not both.) Notably, reconciling the reported data by taking the
minimum (left panel) or mean (right panel) across reported wage values
for those individuals that report both do not lead to meaningful
differences in the distribution, visually. However, reconciling with the
max (middle panel) value leads to slightly less dramatic declines in
accepted wage ratios than in the other two cases. In the following
sections, we proceed with the method that reconciles mutliple reported
values using the minimum value of the wage ratio.

\begin{center}\includegraphics[width=0.9\linewidth]{behav_params_overview_files/figure-latex/disc_descriptives3-1} \end{center}

\begin{center}\includegraphics[width=0.9\linewidth]{behav_params_overview_files/figure-latex/disc_descriptives4-1} \end{center}

Next, we fit a linear and spline fit to the scatted plot of the wage
ratio to unemployment duration before employing any regressions. These
plots both visually indicate a decline in the wage ratio with
unemployment duratio, with the spline fit indicating a potentially
non-linear fit (not yet accounting for selection effects).

\begin{center}\includegraphics[width=0.9\linewidth]{behav_params_overview_files/figure-latex/disc_spline-1} \end{center}

\FloatBarrier

\section{Regressions (non-uniform
sample)}\label{regressions-non-uniform-sample}

Next, before correcting for the non-uniformity of the sample (i.e., that
there are less observations present for higher unemployment durations),
we employ the following cross-sectional econometric specifications (with
various modifications to sample and control variables).\\

\(W_{i} = \alpha_{i} + \beta_{1} d_{i} + \beta_{2}UI_{i} + \beta_{3}X_{i} + \epsilon_{i}\)

where \(W_{i}\): Ratio of accepted wage to wage at lost job (hourly
values).

\(d_{i}\): Unemployment duration in continuous (months) or binned
values.

\(UI_{i}\): Control variable for having used or exhausted unemployment
benefits.

\(X_{i}\): Vector of control variables (sex, age, race, marital status,
education level, and previous wage level).

We present and compare \textasciitilde72 variations on the above model
present with all combinations of the following:

\begin{itemize}
\item
  \textbf{Continuous vs.~Discrete Treatment Variable (2 alternatives):}
  Continuous (monthly) versus binned unemployment duration.
\item
  \textbf{Linear vs.~Quadratic vs.~Cubic representation of the principal
  treatment variable (3 alternatives):} We allow the treatment variable
  to enter non-linearly by testing the presence of quadratic or cubic
  relationships (with the lower-order transformations entering in all
  models). We do not include any non-linear representation of the binned
  unemployment duration as the bins are uneven and would thus require
  additional assumptions for validity.
\item
  \textbf{w. UI vs w. Exhausted UI (3 alternatives):} The survey
  includes a variable for whether individuals USE and/or EXHAUST
  unemployment benefits. We run the regressions without these UI
  controls, with the control for having used UI, or with the control for
  having exhausted UI.
\item
  \textbf{w. Controls (2 alternatives):} With or without additional
  demographic controls (sex, age, race, married, education).
\item
  \textbf{w. Wage Level (2 alternatives):} With or without wage level of
  lost job to control for income and the relationship between wage
  levels and the outcome wage ratio itself.
\item
  \textbf{Outlier clipped sample (2 alternatives):} We either remove
  outliers where the wage ratio is within {[}0.25, 2.5{]} and reported
  unemployment duration is below 96 weeks (\textasciitilde{} 2 years),
  or employ the raw sample.
\end{itemize}

In each regression table, we include the full set of coefficients to
allow for examination of the regression coefficients on the controls as
well as the principal variables of interest. In each table, we highlight
\(\beta_{1}\) as this is the main regression coefficient of interest. We
employ the \emph{check\_model()} function from the \emph{performance}
package in R to display visual checks of various model assumptions.

Across all models (except those that include a control for having
exhausted UI benefits) in the tables below we see a consistently
negative coefficient on unemployment duration (\textasciitilde0.5-1
percentage point increase in the wage ratio for each additional month
spent in unemployment). These coefficients are all statistically
significant at the 0.1\% level. Interestingly, this coefficient loses
statistical significance in any model that controls for having exhausted
UI benefits. Otherwise, examining the performance of our model with
continuous unemployment duration, UI use (not exhaustion), all controls,
wage levels, and outlier correction we see that the model performs
passably across various diagnostic tests.

In the sections that follow, we report all regression results in
regression tables. Additionally, we display model diagnostic plots for
the specification with continuous unemployment duration, UI control,
demographic controls, using the clipped sample, and assuming linearity
in the relationship between unemployment duration and the accepted
re-employment wage ratio. The quantile-quantile plots below reveal that
residuals are approximately normally distributed, though there is
evidence of heavy-tailed behavior in the upper quantiles.

\FloatBarrier

\subsection{Continuous UE Duration}\label{continuous-ue-duration}

Continuous UE duration treatment is reported in monthly values. A
one-unit increase in the treatment variable = 1 additional month of
unemployment.

\begin{table}
\centering\centering
\caption{\label{tab:disc_reg2}Continuous UE Duration w.o Wage Level Control (Clipped Sample)}
\centering
\resizebox{\ifdim\width>\linewidth\linewidth\else\width\fi}{!}{
\begin{tabular}[t]{lcccccccccccc}
\toprule
  & Cont. (clipped) & Cont. w. UI (clipped) & Cont. w. exhausted UI (clipped) & Cont. Sq (clipped) & Cont. Sq w. UI (clipped) & Cont. Sq w. exhausted UI (clipped) & Cont. w. controls (clipped) & Cont. w. UI w. controls (clipped) & Cont. w. exhausted UI w. controls (clipped) & Cont. Sq w. controls (clipped) & Cont. Sq w. UI w. controls (clipped) & Cont. Sq w. exhausted UI w. controls (clipped)\\
\midrule
Intercept & \num{1.045}*** & \num{1.045}*** & \num{1.006}*** & \num{1.046}*** & \num{1.046}*** & \num{1.001}*** & \num{1.163}*** & \num{1.163}*** & \num{1.113}*** & \num{1.163}*** & \num{1.163}*** & \num{1.108}***\\
 & (\num{0.004}) & (\num{0.004}) & (\num{0.007}) & (\num{0.005}) & (\num{0.005}) & (\num{0.008}) & (\num{0.021}) & (\num{0.021}) & (\num{0.023}) & (\num{0.021}) & (\num{0.021}) & (\num{0.023})\\
\cellcolor{yellow!20}{Unemployment Duration (Months)} & \cellcolor{yellow!20}{\num{-0.006}***} & \cellcolor{yellow!20}{\num{-0.006}***} & \cellcolor{yellow!20}{\num{-0.004}***} & \cellcolor{yellow!20}{\num{-0.007}**} & \cellcolor{yellow!20}{\num{-0.007}**} & \cellcolor{yellow!20}{\num{-0.001}} & \cellcolor{yellow!20}{\num{-0.006}***} & \cellcolor{yellow!20}{\num{-0.006}***} & \cellcolor{yellow!20}{\num{-0.004}***} & \cellcolor{yellow!20}{\num{-0.006}**} & \cellcolor{yellow!20}{\num{-0.006}**} & \cellcolor{yellow!20}{\num{-0.001}}\\
 & (\num{0.001}) & (\num{0.001}) & (\num{0.001}) & (\num{0.002}) & (\num{0.002}) & (\num{0.002}) & (\num{0.001}) & (\num{0.001}) & (\num{0.001}) & (\num{0.002}) & (\num{0.002}) & (\num{0.002})\\
Received Unemployment Compensation &  & \num{0.000} &  &  & \num{0.000} &  &  & \num{0.000} &  &  & \num{0.000} & \\
 &  & (\num{0.001}) &  &  & (\num{0.001}) &  &  & (\num{0.001}) &  &  & (\num{0.001}) & \\
Exhausted Unemployment Compensation &  &  & \num{0.001}*** &  &  & \num{0.001}*** &  &  & \num{0.000}*** &  &  & \num{0.001}***\\
 &  &  & (\num{0.000}) &  &  & (\num{0.000}) &  &  & (\num{0.000}) &  &  & (\num{0.000})\\
Unemployment Duration (Months²) &  &  &  & \num{0.000} & \num{0.000} & \num{-0.000} &  &  &  & \num{0.000} & \num{0.000} & \num{-0.000}\\
 &  &  &  & (\num{0.000}) & (\num{0.000}) & (\num{0.000}) &  &  &  & (\num{0.000}) & (\num{0.000}) & (\num{0.000})\\
Female &  &  &  &  &  &  & \num{-0.003} & \num{-0.003} & \num{-0.003} & \num{-0.003} & \num{-0.003} & \num{-0.003}\\
 &  &  &  &  &  &  & (\num{0.007}) & (\num{0.007}) & (\num{0.007}) & (\num{0.007}) & (\num{0.007}) & \vphantom{1} (\num{0.007})\\
Age &  &  &  &  &  &  & \num{-0.002}*** & \num{-0.002}*** & \num{-0.002}*** & \num{-0.002}*** & \num{-0.002}*** & \num{-0.002}***\\
 &  &  &  &  &  &  & (\num{0.000}) & (\num{0.000}) & (\num{0.000}) & (\num{0.000}) & (\num{0.000}) & (\num{0.000})\\
White &  &  &  &  &  &  & \num{-0.052}** & \num{-0.052}** & \num{-0.051}** & \num{-0.052}** & \num{-0.052}** & \num{-0.051}**\\
 &  &  &  &  &  &  & (\num{0.016}) & (\num{0.016}) & (\num{0.016}) & (\num{0.016}) & (\num{0.016}) & (\num{0.016})\\
Black &  &  &  &  &  &  & \num{-0.057}** & \num{-0.057}** & \num{-0.055}** & \num{-0.057}** & \num{-0.057}** & \num{-0.056}**\\
 &  &  &  &  &  &  & (\num{0.018}) & (\num{0.018}) & (\num{0.018}) & (\num{0.018}) & (\num{0.018}) & (\num{0.018})\\
Mixed &  &  &  &  &  &  & \num{-0.070}** & \num{-0.070}* & \num{-0.068}* & \num{-0.070}** & \num{-0.070}* & \num{-0.068}*\\
 &  &  &  &  &  &  & (\num{0.027}) & (\num{0.027}) & (\num{0.027}) & (\num{0.027}) & (\num{0.027}) & (\num{0.027})\\
Married &  &  &  &  &  &  & \num{0.011} & \num{0.011} & \num{0.012}+ & \num{0.011} & \num{0.011} & \num{0.012}+\\
 &  &  &  &  &  &  & (\num{0.007}) & (\num{0.007}) & (\num{0.007}) & (\num{0.007}) & (\num{0.007}) & (\num{0.007})\\
High School &  &  &  &  &  &  & \num{0.001} & \num{0.001} & \num{0.005} & \num{0.001} & \num{0.001} & \num{0.005}\\
 &  &  &  &  &  &  & (\num{0.011}) & (\num{0.011}) & (\num{0.011}) & (\num{0.011}) & (\num{0.011}) & (\num{0.011})\\
Associate's Degree &  &  &  &  &  &  & \num{-0.009} & \num{-0.009} & \num{-0.005} & \num{-0.009} & \num{-0.009} & \num{-0.005}\\
 &  &  &  &  &  &  & (\num{0.014}) & (\num{0.014}) & (\num{0.014}) & (\num{0.014}) & (\num{0.014}) & (\num{0.014})\\
Bachelor's Degree &  &  &  &  &  &  & \num{0.066}*** & \num{0.066}*** & \num{0.070}*** & \num{0.066}*** & \num{0.066}*** & \num{0.070}***\\
 &  &  &  &  &  &  & (\num{0.015}) & (\num{0.015}) & (\num{0.015}) & (\num{0.015}) & (\num{0.015}) & (\num{0.015})\\
Postgraduate Degree &  &  &  &  &  &  & \num{0.030} & \num{0.030} & \num{0.037} & \num{0.030} & \num{0.030} & \num{0.037}\\
 &  &  &  &  &  &  & (\num{0.031}) & (\num{0.031}) & (\num{0.031}) & (\num{0.031}) & (\num{0.031}) & (\num{0.031})\\
\midrule
Num.Obs. & \num{4644} & \num{4644} & \num{4644} & \num{4644} & \num{4644} & \num{4644} & \num{4644} & \num{4644} & \num{4644} & \num{4644} & \num{4644} & \num{4644}\\
R2 & \num{0.012} & \num{0.012} & \num{0.022} & \num{0.012} & \num{0.012} & \num{0.023} & \num{0.032} & \num{0.032} & \num{0.040} & \num{0.032} & \num{0.032} & \num{0.040}\\
R2 Adj. & \num{0.012} & \num{0.011} & \num{0.022} & \num{0.011} & \num{0.011} & \num{0.022} & \num{0.029} & \num{0.029} & \num{0.037} & \num{0.029} & \num{0.029} & \num{0.038}\\
RMSE & \num{0.24} & \num{0.24} & \num{0.24} & \num{0.24} & \num{0.24} & \num{0.24} & \num{0.24} & \num{0.24} & \num{0.24} & \num{0.24} & \num{0.24} & \num{0.24}\\
\bottomrule
\multicolumn{13}{l}{\rule{0pt}{1em}+ p $<$ 0.1, * p $<$ 0.05, ** p $<$ 0.01, *** p $<$ 0.001}\\
\end{tabular}}
\end{table}

\begin{table}
\centering\centering
\caption{\label{tab:disc_reg2}Continuous UE Duration w.o Wage Level Control (Full Sample)}
\centering
\resizebox{\ifdim\width>\linewidth\linewidth\else\width\fi}{!}{
\begin{tabular}[t]{lcccccccccccc}
\toprule
  & Cont. & Cont. w. UI & Cont. w. exhausted UI & Cont. Sq & Cont. Sq w. UI & Cont. Sq w. exhausted UI & Cont. w. controls & Cont. w. UI w. controls & Cont. w. exhausted UI w. controls & Cont. Sq w. controls & Cont. Sq w. UI w. controls & Cont. Sq w. exhausted UI w. controls\\
\midrule
Intercept & \num{1.053}*** & \num{1.053}*** & \num{1.006}*** & \num{1.055}*** & \num{1.055}*** & \num{1.002}*** & \num{1.180}*** & \num{1.180}*** & \num{1.119}*** & \num{1.180}*** & \num{1.180}*** & \num{1.116}***\\
 & (\num{0.006}) & (\num{0.006}) & (\num{0.010}) & (\num{0.007}) & (\num{0.007}) & (\num{0.011}) & (\num{0.031}) & (\num{0.031}) & (\num{0.033}) & (\num{0.031}) & (\num{0.031}) & (\num{0.033})\\
\cellcolor{yellow!20}{Unemployment Duration (Months)} & \cellcolor{yellow!20}{\num{-0.007}***} & \cellcolor{yellow!20}{\num{-0.007}***} & \cellcolor{yellow!20}{\num{-0.005}***} & \cellcolor{yellow!20}{\num{-0.009}***} & \cellcolor{yellow!20}{\num{-0.009}***} & \cellcolor{yellow!20}{\num{-0.003}} & \cellcolor{yellow!20}{\num{-0.006}***} & \cellcolor{yellow!20}{\num{-0.006}***} & \cellcolor{yellow!20}{\num{-0.004}***} & \cellcolor{yellow!20}{\num{-0.008}**} & \cellcolor{yellow!20}{\num{-0.008}**} & \cellcolor{yellow!20}{\num{-0.003}}\\
 & (\num{0.001}) & (\num{0.001}) & (\num{0.001}) & (\num{0.002}) & (\num{0.002}) & (\num{0.003}) & (\num{0.001}) & (\num{0.001}) & (\num{0.001}) & (\num{0.002}) & (\num{0.002}) & (\num{0.003})\\
Received Unemployment Compensation &  & \num{-0.000} &  &  & \num{-0.000} &  &  & \num{0.000} &  &  & \num{0.000} & \\
 &  & (\num{0.001}) &  &  & (\num{0.001}) &  &  & (\num{0.001}) &  &  & (\num{0.001}) & \\
Exhausted Unemployment Compensation &  &  & \num{0.001}*** &  &  & \num{0.001}*** &  &  & \num{0.001}*** &  &  & \num{0.001}***\\
 &  &  & (\num{0.000}) &  &  & (\num{0.000}) &  &  & (\num{0.000}) &  &  & (\num{0.000})\\
Unemployment Duration (Months²) &  &  &  & \num{0.000} & \num{0.000} & \num{-0.000} &  &  &  & \num{0.000} & \num{0.000} & \num{-0.000}\\
 &  &  &  & (\num{0.000}) & (\num{0.000}) & (\num{0.000}) &  &  &  & (\num{0.000}) & (\num{0.000}) & (\num{0.000})\\
Female &  &  &  &  &  &  & \num{0.003} & \num{0.003} & \num{0.003} & \num{0.003} & \num{0.003} & \num{0.003}\\
 &  &  &  &  &  &  & (\num{0.011}) & (\num{0.011}) & (\num{0.011}) & (\num{0.011}) & (\num{0.011}) & \vphantom{1} (\num{0.011})\\
Age &  &  &  &  &  &  & \num{-0.003}*** & \num{-0.003}*** & \num{-0.003}*** & \num{-0.003}*** & \num{-0.003}*** & \num{-0.003}***\\
 &  &  &  &  &  &  & (\num{0.000}) & (\num{0.000}) & (\num{0.000}) & (\num{0.000}) & (\num{0.000}) & (\num{0.000})\\
White &  &  &  &  &  &  & \num{-0.035} & \num{-0.035} & \num{-0.033} & \num{-0.035} & \num{-0.035} & \num{-0.033}\\
 &  &  &  &  &  &  & (\num{0.023}) & (\num{0.023}) & (\num{0.023}) & (\num{0.023}) & (\num{0.023}) & (\num{0.023})\\
Black &  &  &  &  &  &  & \num{-0.048}+ & \num{-0.048}+ & \num{-0.045}+ & \num{-0.048}+ & \num{-0.048}+ & \num{-0.045}+\\
 &  &  &  &  &  &  & (\num{0.026}) & (\num{0.026}) & (\num{0.026}) & (\num{0.026}) & (\num{0.026}) & (\num{0.026})\\
Mixed &  &  &  &  &  &  & \num{0.014} & \num{0.014} & \num{0.017} & \num{0.014} & \num{0.014} & \num{0.016}\\
 &  &  &  &  &  &  & (\num{0.040}) & (\num{0.040}) & (\num{0.040}) & (\num{0.040}) & (\num{0.040}) & (\num{0.040})\\
Married &  &  &  &  &  &  & \num{0.005} & \num{0.005} & \num{0.005} & \num{0.005} & \num{0.005} & \num{0.005}\\
 &  &  &  &  &  &  & (\num{0.011}) & (\num{0.011}) & (\num{0.011}) & (\num{0.011}) & (\num{0.011}) & (\num{0.011})\\
High School &  &  &  &  &  &  & \num{0.005} & \num{0.005} & \num{0.011} & \num{0.006} & \num{0.006} & \num{0.011}\\
 &  &  &  &  &  &  & (\num{0.016}) & (\num{0.016}) & (\num{0.016}) & (\num{0.016}) & (\num{0.016}) & (\num{0.016})\\
Associate's Degree &  &  &  &  &  &  & \num{0.032} & \num{0.032} & \num{0.038}+ & \num{0.032} & \num{0.032} & \num{0.037}+\\
 &  &  &  &  &  &  & (\num{0.021}) & (\num{0.021}) & (\num{0.021}) & (\num{0.021}) & (\num{0.021}) & \vphantom{1} (\num{0.021})\\
Bachelor's Degree &  &  &  &  &  &  & \num{0.079}*** & \num{0.079}*** & \num{0.085}*** & \num{0.080}*** & \num{0.080}*** & \num{0.084}***\\
 &  &  &  &  &  &  & (\num{0.021}) & (\num{0.021}) & (\num{0.021}) & (\num{0.021}) & (\num{0.021}) & (\num{0.021})\\
Postgraduate Degree &  &  &  &  &  &  & \num{0.114}* & \num{0.114}* & \num{0.122}** & \num{0.115}* & \num{0.115}* & \num{0.122}**\\
 &  &  &  &  &  &  & (\num{0.045}) & (\num{0.045}) & (\num{0.045}) & (\num{0.045}) & (\num{0.045}) & (\num{0.045})\\
\midrule
Num.Obs. & \num{4870} & \num{4870} & \num{4870} & \num{4870} & \num{4870} & \num{4870} & \num{4870} & \num{4870} & \num{4870} & \num{4870} & \num{4870} & \num{4870}\\
R2 & \num{0.009} & \num{0.009} & \num{0.017} & \num{0.010} & \num{0.010} & \num{0.017} & \num{0.025} & \num{0.025} & \num{0.030} & \num{0.025} & \num{0.025} & \num{0.030}\\
R2 Adj. & \num{0.009} & \num{0.009} & \num{0.016} & \num{0.009} & \num{0.009} & \num{0.016} & \num{0.022} & \num{0.022} & \num{0.028} & \num{0.022} & \num{0.022} & \num{0.027}\\
F & \num{46.344} & \num{23.169} & \num{41.487} & \num{23.546} & \num{15.694} & \num{27.802} & \num{11.151} & \num{10.220} & \num{12.521} & \num{10.252} & \num{9.462} & \num{11.589}\\
RMSE & \num{0.38} & \num{0.38} & \num{0.37} & \num{0.38} & \num{0.38} & \num{0.37} & \num{0.37} & \num{0.37} & \num{0.37} & \num{0.37} & \num{0.37} & \num{0.37}\\
\bottomrule
\multicolumn{13}{l}{\rule{0pt}{1em}+ p $<$ 0.1, * p $<$ 0.05, ** p $<$ 0.01, *** p $<$ 0.001}\\
\end{tabular}}
\end{table}

\begin{center}\includegraphics[width=0.9\linewidth]{behav_params_overview_files/figure-latex/disc_reg1-1} \end{center}

\begin{center}\includegraphics[width=0.9\linewidth]{behav_params_overview_files/figure-latex/disc_reg3-1} \end{center}

\begin{table}
\centering\centering
\caption{\label{tab:disc_reg4}Continuous UE Duration w. Wage Level Control (Clipped Sample)}
\centering
\resizebox{\ifdim\width>\linewidth\linewidth\else\width\fi}{!}{
\begin{tabular}[t]{lcccccccccccc}
\toprule
  & Cont. (clipped) & Cont. w. UI (clipped) & Cont. w. exhausted UI (clipped) & Cont. Sq (clipped) & Cont. Sq w. UI (clipped) & Cont. Sq w. exhausted UI (clipped) & Cont. w. controls (clipped) & Cont. w. UI w. controls (clipped) & Cont. w. exhausted UI w. controls (clipped) & Cont. Sq w. controls (clipped) & Cont. Sq w. UI w. controls (clipped) & Cont. Sq w. exhausted UI w. controls (clipped)\\
\midrule
Intercept & \num{1.131}*** & \num{1.130}*** & \num{1.094}*** & \num{1.131}*** & \num{1.131}*** & \num{1.090}*** & \num{1.217}*** & \num{1.217}*** & \num{1.173}*** & \num{1.217}*** & \num{1.217}*** & \num{1.169}***\\
 & (\num{0.008}) & (\num{0.008}) & (\num{0.010}) & (\num{0.008}) & (\num{0.008}) & (\num{0.011}) & (\num{0.021}) & (\num{0.021}) & (\num{0.022}) & (\num{0.021}) & (\num{0.021}) & (\num{0.023})\\
\cellcolor{yellow!20}{Hourly Wage of Lost Job} & \cellcolor{yellow!20}{\num{-0.006}***} & \cellcolor{yellow!20}{\num{-0.006}***} & \cellcolor{yellow!20}{\num{-0.006}***} & \cellcolor{yellow!20}{\num{-0.006}***} & \cellcolor{yellow!20}{\num{-0.006}***} & \cellcolor{yellow!20}{\num{-0.006}***} & \cellcolor{yellow!20}{\num{-0.007}***} & \cellcolor{yellow!20}{\num{-0.007}***} & \cellcolor{yellow!20}{\num{-0.007}***} & \cellcolor{yellow!20}{\num{-0.007}***} & \cellcolor{yellow!20}{\num{-0.007}***} & \cellcolor{yellow!20}{\num{-0.007}***}\\
 & (\num{0.000}) & (\num{0.000}) & (\num{0.000}) & (\num{0.000}) & (\num{0.000}) & (\num{0.000}) & (\num{0.000}) & (\num{0.000}) & (\num{0.000}) & (\num{0.000}) & (\num{0.000}) & (\num{0.000})\\
\cellcolor{yellow!20}{Unemployment Duration (Months)} & \cellcolor{yellow!20}{\num{-0.006}***} & \cellcolor{yellow!20}{\num{-0.006}***} & \cellcolor{yellow!20}{\num{-0.004}***} & \cellcolor{yellow!20}{\num{-0.006}**} & \cellcolor{yellow!20}{\num{-0.006}**} & \cellcolor{yellow!20}{\num{-0.002}} & \cellcolor{yellow!20}{\num{-0.006}***} & \cellcolor{yellow!20}{\num{-0.006}***} & \cellcolor{yellow!20}{\num{-0.004}***} & \cellcolor{yellow!20}{\num{-0.006}**} & \cellcolor{yellow!20}{\num{-0.006}**} & \cellcolor{yellow!20}{\num{-0.002}}\\
 & (\num{0.001}) & (\num{0.001}) & (\num{0.001}) & (\num{0.002}) & (\num{0.002}) & (\num{0.002}) & (\num{0.001}) & (\num{0.001}) & (\num{0.001}) & (\num{0.002}) & (\num{0.002}) & (\num{0.002})\\
Received Unemployment Compensation &  & \num{0.000} &  &  & \num{0.000} &  &  & \num{0.000} &  &  & \num{0.000} & \\
 &  & (\num{0.001}) &  &  & (\num{0.001}) &  &  & (\num{0.001}) &  &  & (\num{0.001}) & \\
Exhausted Unemployment Compensation &  &  & \num{0.000}*** &  &  & \num{0.000}*** &  &  & \num{0.000}*** &  &  & \num{0.000}***\\
 &  &  & (\num{0.000}) &  &  & (\num{0.000}) &  &  & (\num{0.000}) &  &  & (\num{0.000})\\
Unemployment Duration (Months²) &  &  &  & \num{0.000} & \num{0.000} & \num{-0.000} &  &  &  & \num{0.000} & \num{0.000} & \num{-0.000}\\
 &  &  &  & (\num{0.000}) & (\num{0.000}) & (\num{0.000}) &  &  &  & (\num{0.000}) & (\num{0.000}) & (\num{0.000})\\
Female &  &  &  &  &  &  & \num{-0.023}** & \num{-0.023}** & \num{-0.023}** & \num{-0.023}** & \num{-0.023}** & \num{-0.023}**\\
 &  &  &  &  &  &  & (\num{0.007}) & (\num{0.007}) & (\num{0.007}) & (\num{0.007}) & (\num{0.007}) & \vphantom{1} (\num{0.007})\\
Age &  &  &  &  &  &  & \num{-0.001}*** & \num{-0.001}*** & \num{-0.001}*** & \num{-0.001}*** & \num{-0.001}*** & \num{-0.001}***\\
 &  &  &  &  &  &  & (\num{0.000}) & (\num{0.000}) & (\num{0.000}) & (\num{0.000}) & (\num{0.000}) & (\num{0.000})\\
White &  &  &  &  &  &  & \num{-0.050}** & \num{-0.050}** & \num{-0.049}** & \num{-0.050}** & \num{-0.050}** & \num{-0.050}**\\
 &  &  &  &  &  &  & (\num{0.016}) & (\num{0.016}) & (\num{0.016}) & (\num{0.016}) & (\num{0.016}) & (\num{0.016})\\
Black &  &  &  &  &  &  & \num{-0.061}*** & \num{-0.061}*** & \num{-0.060}*** & \num{-0.061}*** & \num{-0.061}*** & \num{-0.060}***\\
 &  &  &  &  &  &  & (\num{0.018}) & (\num{0.018}) & (\num{0.018}) & (\num{0.018}) & (\num{0.018}) & (\num{0.018})\\
Mixed &  &  &  &  &  &  & \num{-0.067}* & \num{-0.067}* & \num{-0.065}* & \num{-0.067}* & \num{-0.067}* & \num{-0.066}*\\
 &  &  &  &  &  &  & (\num{0.027}) & (\num{0.027}) & (\num{0.026}) & (\num{0.027}) & (\num{0.027}) & (\num{0.026})\\
Married &  &  &  &  &  &  & \num{0.018}* & \num{0.018}* & \num{0.018}* & \num{0.018}* & \num{0.018}* & \num{0.018}*\\
 &  &  &  &  &  &  & (\num{0.007}) & (\num{0.007}) & (\num{0.007}) & (\num{0.007}) & (\num{0.007}) & (\num{0.007})\\
High School &  &  &  &  &  &  & \num{0.019}+ & \num{0.019}+ & \num{0.022}* & \num{0.019}+ & \num{0.019}+ & \num{0.022}*\\
 &  &  &  &  &  &  & (\num{0.011}) & (\num{0.011}) & (\num{0.011}) & (\num{0.011}) & (\num{0.011}) & (\num{0.011})\\
Associate's Degree &  &  &  &  &  &  & \num{0.027}+ & \num{0.027}+ & \num{0.029}* & \num{0.027}+ & \num{0.027}+ & \num{0.030}*\\
 &  &  &  &  &  &  & (\num{0.014}) & (\num{0.014}) & (\num{0.014}) & (\num{0.014}) & (\num{0.014}) & (\num{0.014})\\
Bachelor's Degree &  &  &  &  &  &  & \num{0.121}*** & \num{0.121}*** & \num{0.123}*** & \num{0.121}*** & \num{0.121}*** & \num{0.123}***\\
 &  &  &  &  &  &  & (\num{0.015}) & (\num{0.015}) & (\num{0.015}) & (\num{0.015}) & (\num{0.015}) & (\num{0.015})\\
Postgraduate Degree &  &  &  &  &  &  & \num{0.120}*** & \num{0.120}*** & \num{0.123}*** & \num{0.120}*** & \num{0.120}*** & \num{0.123}***\\
 &  &  &  &  &  &  & (\num{0.031}) & (\num{0.031}) & (\num{0.031}) & (\num{0.031}) & (\num{0.031}) & (\num{0.031})\\
\midrule
Num.Obs. & \num{4644} & \num{4644} & \num{4644} & \num{4644} & \num{4644} & \num{4644} & \num{4644} & \num{4644} & \num{4644} & \num{4644} & \num{4644} & \num{4644}\\
R2 & \num{0.046} & \num{0.046} & \num{0.053} & \num{0.046} & \num{0.046} & \num{0.053} & \num{0.073} & \num{0.073} & \num{0.079} & \num{0.073} & \num{0.073} & \num{0.079}\\
R2 Adj. & \num{0.046} & \num{0.046} & \num{0.053} & \num{0.046} & \num{0.045} & \num{0.053} & \num{0.071} & \num{0.071} & \num{0.077} & \num{0.071} & \num{0.071} & \num{0.077}\\
RMSE & \num{0.24} & \num{0.24} & \num{0.24} & \num{0.24} & \num{0.24} & \num{0.24} & \num{0.23} & \num{0.23} & \num{0.23} & \num{0.23} & \num{0.23} & \num{0.23}\\
\bottomrule
\multicolumn{13}{l}{\rule{0pt}{1em}+ p $<$ 0.1, * p $<$ 0.05, ** p $<$ 0.01, *** p $<$ 0.001}\\
\end{tabular}}
\end{table}

\begin{table}
\centering\centering
\caption{\label{tab:disc_reg4}Continuous UE Duration w. Wage Level Control (Full Sample)}
\centering
\resizebox{\ifdim\width>\linewidth\linewidth\else\width\fi}{!}{
\begin{tabular}[t]{lcccccccccccc}
\toprule
  & Cont. & Cont. w. UI & Cont. w. exhausted UI & Cont. Sq & Cont. Sq w. UI & Cont. Sq w. exhausted UI & Cont. w. controls & Cont. w. UI w. controls & Cont. w. exhausted UI w. controls & Cont. Sq w. controls & Cont. Sq w. UI w. controls & Cont. Sq w. exhausted UI w. controls\\
\midrule
Intercept & \num{1.185}*** & \num{1.186}*** & \num{1.145}*** & \num{1.186}*** & \num{1.187}*** & \num{1.141}*** & \num{1.263}*** & \num{1.263}*** & \num{1.213}*** & \num{1.263}*** & \num{1.263}*** & \num{1.210}***\\
 & (\num{0.011}) & (\num{0.011}) & (\num{0.014}) & (\num{0.012}) & (\num{0.012}) & (\num{0.015}) & (\num{0.031}) & (\num{0.031}) & (\num{0.033}) & (\num{0.031}) & (\num{0.031}) & (\num{0.033})\\
\cellcolor{yellow!20}{Hourly Wage of Lost Job} & \cellcolor{yellow!20}{\num{-0.009}***} & \cellcolor{yellow!20}{\num{-0.009}***} & \cellcolor{yellow!20}{\num{-0.009}***} & \cellcolor{yellow!20}{\num{-0.009}***} & \cellcolor{yellow!20}{\num{-0.009}***} & \cellcolor{yellow!20}{\num{-0.009}***} & \cellcolor{yellow!20}{\num{-0.011}***} & \cellcolor{yellow!20}{\num{-0.011}***} & \cellcolor{yellow!20}{\num{-0.011}***} & \cellcolor{yellow!20}{\num{-0.011}***} & \cellcolor{yellow!20}{\num{-0.011}***} & \cellcolor{yellow!20}{\num{-0.011}***}\\
 & (\num{0.001}) & (\num{0.001}) & (\num{0.001}) & (\num{0.001}) & (\num{0.001}) & (\num{0.001}) & (\num{0.001}) & (\num{0.001}) & (\num{0.001}) & (\num{0.001}) & (\num{0.001}) & (\num{0.001})\\
\cellcolor{yellow!20}{Unemployment Duration (Months)} & \cellcolor{yellow!20}{\num{-0.007}***} & \cellcolor{yellow!20}{\num{-0.007}***} & \cellcolor{yellow!20}{\num{-0.005}***} & \cellcolor{yellow!20}{\num{-0.007}**} & \cellcolor{yellow!20}{\num{-0.007}**} & \cellcolor{yellow!20}{\num{-0.003}} & \cellcolor{yellow!20}{\num{-0.006}***} & \cellcolor{yellow!20}{\num{-0.006}***} & \cellcolor{yellow!20}{\num{-0.005}***} & \cellcolor{yellow!20}{\num{-0.007}**} & \cellcolor{yellow!20}{\num{-0.007}**} & \cellcolor{yellow!20}{\num{-0.003}}\\
 & (\num{0.001}) & (\num{0.001}) & (\num{0.001}) & (\num{0.002}) & (\num{0.002}) & (\num{0.003}) & (\num{0.001}) & (\num{0.001}) & (\num{0.001}) & (\num{0.002}) & (\num{0.002}) & (\num{0.003})\\
Received Unemployment Compensation &  & \num{-0.000} &  &  & \num{-0.000} &  &  & \num{-0.000} &  &  & \num{-0.000} & \\
 &  & (\num{0.001}) &  &  & (\num{0.001}) &  &  & (\num{0.001}) &  &  & (\num{0.001}) & \\
Exhausted Unemployment Compensation &  &  & \num{0.001}*** &  &  & \num{0.001}*** &  &  & \num{0.000}*** &  &  & \num{0.000}***\\
 &  &  & (\num{0.000}) &  &  & (\num{0.000}) &  &  & (\num{0.000}) &  &  & (\num{0.000})\\
Unemployment Duration (Months²) &  &  &  & \num{0.000} & \num{0.000} & \num{-0.000} &  &  &  & \num{0.000} & \num{0.000} & \num{-0.000}\\
 &  &  &  & (\num{0.000}) & (\num{0.000}) & (\num{0.000}) &  &  &  & (\num{0.000}) & (\num{0.000}) & (\num{0.000})\\
Female &  &  &  &  &  &  & \num{-0.028}** & \num{-0.028}** & \num{-0.028}** & \num{-0.028}** & \num{-0.028}** & \num{-0.028}**\\
 &  &  &  &  &  &  & (\num{0.011}) & (\num{0.011}) & (\num{0.011}) & (\num{0.011}) & (\num{0.011}) & (\num{0.011})\\
Age &  &  &  &  &  &  & \num{-0.002}*** & \num{-0.002}*** & \num{-0.001}*** & \num{-0.002}*** & \num{-0.002}*** & \num{-0.001}***\\
 &  &  &  &  &  &  & (\num{0.000}) & (\num{0.000}) & (\num{0.000}) & (\num{0.000}) & (\num{0.000}) & (\num{0.000})\\
White &  &  &  &  &  &  & \num{-0.034} & \num{-0.034} & \num{-0.032} & \num{-0.034} & \num{-0.034} & \num{-0.032}\\
 &  &  &  &  &  &  & (\num{0.023}) & (\num{0.023}) & (\num{0.023}) & (\num{0.023}) & (\num{0.023}) & (\num{0.023})\\
Black &  &  &  &  &  &  & \num{-0.058}* & \num{-0.058}* & \num{-0.055}* & \num{-0.057}* & \num{-0.057}* & \num{-0.055}*\\
 &  &  &  &  &  &  & (\num{0.026}) & (\num{0.026}) & (\num{0.026}) & (\num{0.026}) & (\num{0.026}) & (\num{0.026})\\
Mixed &  &  &  &  &  &  & \num{0.016} & \num{0.016} & \num{0.019} & \num{0.016} & \num{0.016} & \num{0.018}\\
 &  &  &  &  &  &  & (\num{0.039}) & (\num{0.039}) & (\num{0.039}) & (\num{0.039}) & (\num{0.039}) & (\num{0.039})\\
Married &  &  &  &  &  &  & \num{0.013} & \num{0.013} & \num{0.013} & \num{0.013} & \num{0.013} & \num{0.014}\\
 &  &  &  &  &  &  & (\num{0.010}) & (\num{0.010}) & (\num{0.010}) & (\num{0.010}) & (\num{0.010}) & (\num{0.010})\\
High School &  &  &  &  &  &  & \num{0.033}* & \num{0.033}* & \num{0.037}* & \num{0.033}* & \num{0.033}* & \num{0.037}*\\
 &  &  &  &  &  &  & (\num{0.015}) & (\num{0.015}) & (\num{0.015}) & (\num{0.015}) & (\num{0.015}) & (\num{0.015})\\
Associate's Degree &  &  &  &  &  &  & \num{0.084}*** & \num{0.084}*** & \num{0.088}*** & \num{0.084}*** & \num{0.084}*** & \num{0.087}***\\
 &  &  &  &  &  &  & (\num{0.021}) & (\num{0.021}) & (\num{0.021}) & (\num{0.021}) & (\num{0.021}) & (\num{0.021})\\
Bachelor's Degree &  &  &  &  &  &  & \num{0.161}*** & \num{0.161}*** & \num{0.164}*** & \num{0.161}*** & \num{0.161}*** & \num{0.163}***\\
 &  &  &  &  &  &  & (\num{0.022}) & (\num{0.022}) & (\num{0.022}) & (\num{0.022}) & (\num{0.022}) & (\num{0.022})\\
Postgraduate Degree &  &  &  &  &  &  & \num{0.244}*** & \num{0.244}*** & \num{0.248}*** & \num{0.245}*** & \num{0.245}*** & \num{0.248}***\\
 &  &  &  &  &  &  & (\num{0.045}) & (\num{0.045}) & (\num{0.045}) & (\num{0.045}) & (\num{0.045}) & (\num{0.045})\\
\midrule
Num.Obs. & \num{4870} & \num{4870} & \num{4870} & \num{4870} & \num{4870} & \num{4870} & \num{4870} & \num{4870} & \num{4870} & \num{4870} & \num{4870} & \num{4870}\\
R2 & \num{0.048} & \num{0.048} & \num{0.052} & \num{0.048} & \num{0.048} & \num{0.052} & \num{0.069} & \num{0.069} & \num{0.073} & \num{0.069} & \num{0.069} & \num{0.073}\\
R2 Adj. & \num{0.047} & \num{0.047} & \num{0.051} & \num{0.047} & \num{0.047} & \num{0.051} & \num{0.067} & \num{0.067} & \num{0.070} & \num{0.067} & \num{0.067} & \num{0.070}\\
F & \num{121.551} & \num{81.034} & \num{88.352} & \num{81.047} & \num{60.784} & \num{66.451} & \num{30.216} & \num{27.890} & \num{29.347} & \num{27.893} & \num{25.899} & \num{27.287}\\
RMSE & \num{0.37} & \num{0.37} & \num{0.37} & \num{0.37} & \num{0.37} & \num{0.37} & \num{0.37} & \num{0.37} & \num{0.37} & \num{0.37} & \num{0.37} & \num{0.37}\\
\bottomrule
\multicolumn{13}{l}{\rule{0pt}{1em}+ p $<$ 0.1, * p $<$ 0.05, ** p $<$ 0.01, *** p $<$ 0.001}\\
\end{tabular}}
\end{table}

\FloatBarrier

\subsection{Binned UE Duration}\label{binned-ue-duration}

Binned UE duration treatment is reported in bins as indicated in the box
plots and code cleaning above.

\begin{table}
\centering\centering
\caption{\label{tab:disc_reg5}Binned UE Duration w.o Wage Level Control (Clipped Sample)}
\centering
\resizebox{\ifdim\width>\linewidth\linewidth\else\width\fi}{!}{
\begin{tabular}[t]{lcccccc}
\toprule
  & Disc. (clipped) & Disc. w. UI (clipped) & Disc. w. exhausted UI (clipped) & Disc. w. controls (clipped) & Disc. w. UI w. controls (clipped) & Disc. w. exhausted UI w. controls (clipped)\\
\midrule
Intercept & \num{1.055}*** & \num{1.055}*** & \num{1.010}*** & \num{1.170}*** & \num{1.170}*** & \num{1.116}***\\
 & (\num{0.005}) & (\num{0.005}) & (\num{0.008}) & (\num{0.021}) & (\num{0.021}) & (\num{0.023})\\
\cellcolor{yellow!20}{Unemployment Duration (Binned)} & \cellcolor{yellow!20}{\num{-0.009}***} & \cellcolor{yellow!20}{\num{-0.009}***} & \cellcolor{yellow!20}{\num{-0.005}***} & \cellcolor{yellow!20}{\num{-0.008}***} & \cellcolor{yellow!20}{\num{-0.008}***} & \cellcolor{yellow!20}{\num{-0.005}***}\\
 & (\num{0.001}) & (\num{0.001}) & (\num{0.001}) & (\num{0.001}) & (\num{0.001}) & (\num{0.001})\\
Received Unemployment Compensation &  & \num{0.000} &  &  & \num{0.000} & \\
 &  & (\num{0.001}) &  &  & (\num{0.001}) & \\
Exhausted Unemployment Compensation &  &  & \num{0.001}*** &  &  & \num{0.001}***\\
 &  &  & (\num{0.000}) &  &  & (\num{0.000})\\
Female &  &  &  & \num{-0.003} & \num{-0.003} & \num{-0.003}\\
 &  &  &  & (\num{0.007}) & (\num{0.007}) & \vphantom{1} (\num{0.007})\\
Age &  &  &  & \num{-0.002}*** & \num{-0.002}*** & \num{-0.002}***\\
 &  &  &  & (\num{0.000}) & (\num{0.000}) & (\num{0.000})\\
White &  &  &  & \num{-0.052}** & \num{-0.052}** & \num{-0.050}**\\
 &  &  &  & (\num{0.016}) & (\num{0.016}) & (\num{0.016})\\
Black &  &  &  & \num{-0.056}** & \num{-0.056}** & \num{-0.055}**\\
 &  &  &  & (\num{0.018}) & (\num{0.018}) & (\num{0.018})\\
Mixed &  &  &  & \num{-0.070}** & \num{-0.070}* & \num{-0.068}*\\
 &  &  &  & (\num{0.027}) & (\num{0.027}) & (\num{0.027})\\
Married &  &  &  & \num{0.011} & \num{0.011} & \num{0.012}\\
 &  &  &  & (\num{0.007}) & (\num{0.007}) & (\num{0.007})\\
High School &  &  &  & \num{0.001} & \num{0.001} & \num{0.005}\\
 &  &  &  & (\num{0.011}) & (\num{0.011}) & (\num{0.011})\\
Associate's Degree &  &  &  & \num{-0.009} & \num{-0.009} & \num{-0.005}\\
 &  &  &  & (\num{0.014}) & (\num{0.014}) & (\num{0.014})\\
Bachelor's Degree &  &  &  & \num{0.067}*** & \num{0.067}*** & \num{0.071}***\\
 &  &  &  & (\num{0.015}) & (\num{0.015}) & (\num{0.015})\\
Postgraduate Degree &  &  &  & \num{0.030} & \num{0.030} & \num{0.038}\\
 &  &  &  & (\num{0.031}) & (\num{0.031}) & (\num{0.031})\\
\midrule
Num.Obs. & \num{4644} & \num{4644} & \num{4644} & \num{4644} & \num{4644} & \num{4644}\\
R2 & \num{0.011} & \num{0.011} & \num{0.021} & \num{0.031} & \num{0.031} & \num{0.039}\\
R2 Adj. & \num{0.011} & \num{0.010} & \num{0.021} & \num{0.028} & \num{0.028} & \num{0.036}\\
RMSE & \num{0.24} & \num{0.24} & \num{0.24} & \num{0.24} & \num{0.24} & \num{0.24}\\
\bottomrule
\multicolumn{7}{l}{\rule{0pt}{1em}+ p $<$ 0.1, * p $<$ 0.05, ** p $<$ 0.01, *** p $<$ 0.001}\\
\end{tabular}}
\end{table}

\begin{table}
\centering\centering
\caption{\label{tab:disc_reg5}Binned UE Duration w.o Wage Level Control (Full Sample)}
\centering
\resizebox{\ifdim\width>\linewidth\linewidth\else\width\fi}{!}{
\begin{tabular}[t]{lcccccc}
\toprule
  & Disc. & Disc. w. UI & Disc. w. exhausted UI & Disc. w. controls & Disc. w. UI w. controls & Disc. w. exhausted UI w. controls\\
\midrule
Intercept & \num{1.069}*** & \num{1.069}*** & \num{1.016}*** & \num{1.190}*** & \num{1.190}*** & \num{1.127}***\\
 & (\num{0.008}) & (\num{0.008}) & (\num{0.012}) & (\num{0.031}) & (\num{0.031}) & (\num{0.034})\\
\cellcolor{yellow!20}{Unemployment Duration (Binned)} & \cellcolor{yellow!20}{\num{-0.013}***} & \cellcolor{yellow!20}{\num{-0.013}***} & \cellcolor{yellow!20}{\num{-0.008}***} & \cellcolor{yellow!20}{\num{-0.011}***} & \cellcolor{yellow!20}{\num{-0.011}***} & \cellcolor{yellow!20}{\num{-0.007}***}\\
 & (\num{0.002}) & (\num{0.002}) & (\num{0.002}) & (\num{0.002}) & (\num{0.002}) & (\num{0.002})\\
Received Unemployment Compensation &  & \num{-0.000} &  &  & \num{0.000} & \\
 &  & (\num{0.001}) &  &  & (\num{0.001}) & \\
Exhausted Unemployment Compensation &  &  & \num{0.001}*** &  &  & \num{0.001}***\\
 &  &  & (\num{0.000}) &  &  & (\num{0.000})\\
Female &  &  &  & \num{0.003} & \num{0.003} & \num{0.003}\\
 &  &  &  & (\num{0.011}) & (\num{0.011}) & \vphantom{1} (\num{0.011})\\
Age &  &  &  & \num{-0.003}*** & \num{-0.003}*** & \num{-0.003}***\\
 &  &  &  & (\num{0.000}) & (\num{0.000}) & (\num{0.000})\\
White &  &  &  & \num{-0.035} & \num{-0.035} & \num{-0.033}\\
 &  &  &  & (\num{0.023}) & (\num{0.023}) & (\num{0.023})\\
Black &  &  &  & \num{-0.047}+ & \num{-0.047}+ & \num{-0.045}+\\
 &  &  &  & (\num{0.026}) & (\num{0.026}) & (\num{0.026})\\
Mixed &  &  &  & \num{0.014} & \num{0.014} & \num{0.017}\\
 &  &  &  & (\num{0.040}) & (\num{0.040}) & (\num{0.040})\\
Married &  &  &  & \num{0.004} & \num{0.004} & \num{0.005}\\
 &  &  &  & (\num{0.011}) & (\num{0.011}) & (\num{0.011})\\
High School &  &  &  & \num{0.006} & \num{0.006} & \num{0.012}\\
 &  &  &  & (\num{0.016}) & (\num{0.016}) & (\num{0.016})\\
Associate's Degree &  &  &  & \num{0.033} & \num{0.033} & \num{0.038}+\\
 &  &  &  & (\num{0.021}) & (\num{0.021}) & \vphantom{1} (\num{0.021})\\
Bachelor's Degree &  &  &  & \num{0.082}*** & \num{0.082}*** & \num{0.087}***\\
 &  &  &  & (\num{0.021}) & (\num{0.021}) & (\num{0.021})\\
Postgraduate Degree &  &  &  & \num{0.116}** & \num{0.116}** & \num{0.124}**\\
 &  &  &  & (\num{0.045}) & (\num{0.045}) & (\num{0.045})\\
\midrule
Num.Obs. & \num{4870} & \num{4870} & \num{4870} & \num{4870} & \num{4870} & \num{4870}\\
R2 & \num{0.010} & \num{0.010} & \num{0.016} & \num{0.025} & \num{0.025} & \num{0.030}\\
R2 Adj. & \num{0.009} & \num{0.009} & \num{0.016} & \num{0.022} & \num{0.022} & \num{0.027}\\
F & \num{47.638} & \num{23.816} & \num{40.199} & \num{11.165} & \num{10.232} & \num{12.314}\\
RMSE & \num{0.37} & \num{0.37} & \num{0.37} & \num{0.37} & \num{0.37} & \num{0.37}\\
\bottomrule
\multicolumn{7}{l}{\rule{0pt}{1em}+ p $<$ 0.1, * p $<$ 0.05, ** p $<$ 0.01, *** p $<$ 0.001}\\
\end{tabular}}
\end{table}

\begin{table}
\centering\centering
\caption{\label{tab:disc_reg6}Binned UE Duration w. Wage Level Control (Clipped Sample)}
\centering
\resizebox{\ifdim\width>\linewidth\linewidth\else\width\fi}{!}{
\begin{tabular}[t]{lcccccc}
\toprule
  & Disc. (clipped) & Disc. w. UI (clipped) & Disc. w. exhausted UI (clipped) & Disc. w. controls (clipped) & Disc. w. UI w. controls (clipped) & Disc. w. exhausted UI w. controls (clipped)\\
\midrule
Intercept & \num{1.139}*** & \num{1.139}*** & \num{1.098}*** & \num{1.224}*** & \num{1.224}*** & \num{1.176}***\\
 & (\num{0.008}) & (\num{0.008}) & (\num{0.011}) & (\num{0.021}) & (\num{0.021}) & (\num{0.023})\\
\cellcolor{yellow!20}{Hourly Wage of Lost Job} & \cellcolor{yellow!20}{\num{-0.006}***} & \cellcolor{yellow!20}{\num{-0.006}***} & \cellcolor{yellow!20}{\num{-0.006}***} & \cellcolor{yellow!20}{\num{-0.007}***} & \cellcolor{yellow!20}{\num{-0.007}***} & \cellcolor{yellow!20}{\num{-0.007}***}\\
 & (\num{0.000}) & (\num{0.000}) & (\num{0.000}) & (\num{0.000}) & (\num{0.000}) & (\num{0.000})\\
\cellcolor{yellow!20}{Unemployment Duration (Binned)} & \cellcolor{yellow!20}{\num{-0.009}***} & \cellcolor{yellow!20}{\num{-0.009}***} & \cellcolor{yellow!20}{\num{-0.005}***} & \cellcolor{yellow!20}{\num{-0.008}***} & \cellcolor{yellow!20}{\num{-0.008}***} & \cellcolor{yellow!20}{\num{-0.005}***}\\
 & (\num{0.001}) & (\num{0.001}) & (\num{0.001}) & (\num{0.001}) & (\num{0.001}) & (\num{0.001})\\
Received Unemployment Compensation &  & \num{0.000} &  &  & \num{-0.000} & \\
 &  & (\num{0.001}) &  &  & (\num{0.001}) & \\
Exhausted Unemployment Compensation &  &  & \num{0.000}*** &  &  & \num{0.000}***\\
 &  &  & (\num{0.000}) &  &  & (\num{0.000})\\
Female &  &  &  & \num{-0.023}** & \num{-0.023}** & \num{-0.023}**\\
 &  &  &  & (\num{0.007}) & (\num{0.007}) & \vphantom{1} (\num{0.007})\\
Age &  &  &  & \num{-0.001}*** & \num{-0.001}*** & \num{-0.001}***\\
 &  &  &  & (\num{0.000}) & (\num{0.000}) & (\num{0.000})\\
White &  &  &  & \num{-0.050}** & \num{-0.050}** & \num{-0.049}**\\
 &  &  &  & (\num{0.016}) & (\num{0.016}) & (\num{0.016})\\
Black &  &  &  & \num{-0.061}*** & \num{-0.061}*** & \num{-0.059}***\\
 &  &  &  & (\num{0.018}) & (\num{0.018}) & (\num{0.018})\\
Mixed &  &  &  & \num{-0.067}* & \num{-0.067}* & \num{-0.065}*\\
 &  &  &  & (\num{0.027}) & (\num{0.027}) & (\num{0.026})\\
Married &  &  &  & \num{0.017}* & \num{0.017}* & \num{0.018}*\\
 &  &  &  & (\num{0.007}) & (\num{0.007}) & (\num{0.007})\\
High School &  &  &  & \num{0.019}+ & \num{0.019}+ & \num{0.022}*\\
 &  &  &  & (\num{0.011}) & (\num{0.011}) & (\num{0.011})\\
Associate's Degree &  &  &  & \num{0.027}+ & \num{0.027}+ & \num{0.030}*\\
 &  &  &  & (\num{0.014}) & (\num{0.014}) & (\num{0.014})\\
Bachelor's Degree &  &  &  & \num{0.122}*** & \num{0.122}*** & \num{0.124}***\\
 &  &  &  & (\num{0.015}) & (\num{0.015}) & (\num{0.015})\\
Postgraduate Degree &  &  &  & \num{0.120}*** & \num{0.120}*** & \num{0.124}***\\
 &  &  &  & (\num{0.031}) & (\num{0.031}) & (\num{0.031})\\
\midrule
Num.Obs. & \num{4644} & \num{4644} & \num{4644} & \num{4644} & \num{4644} & \num{4644}\\
R2 & \num{0.045} & \num{0.045} & \num{0.052} & \num{0.072} & \num{0.072} & \num{0.078}\\
R2 Adj. & \num{0.045} & \num{0.045} & \num{0.051} & \num{0.070} & \num{0.070} & \num{0.076}\\
RMSE & \num{0.24} & \num{0.24} & \num{0.24} & \num{0.23} & \num{0.23} & \num{0.23}\\
\bottomrule
\multicolumn{7}{l}{\rule{0pt}{1em}+ p $<$ 0.1, * p $<$ 0.05, ** p $<$ 0.01, *** p $<$ 0.001}\\
\end{tabular}}
\end{table}

\begin{table}
\centering\centering
\caption{\label{tab:disc_reg6}Binned UE Duration w. Wage Level Control (Full Sample)}
\centering
\resizebox{\ifdim\width>\linewidth\linewidth\else\width\fi}{!}{
\begin{tabular}[t]{lcccccc}
\toprule
  & Disc. & Disc. w. UI & Disc. w. exhausted UI & Disc. w. controls & Disc. w. UI w. controls & Disc. w. exhausted UI w. controls\\
\midrule
Intercept & \num{1.198}*** & \num{1.199}*** & \num{1.154}*** & \num{1.272}*** & \num{1.272}*** & \num{1.220}***\\
 & (\num{0.012}) & (\num{0.012}) & (\num{0.016}) & (\num{0.031}) & (\num{0.031}) & (\num{0.034})\\
\cellcolor{yellow!20}{Hourly Wage of Lost Job} & \cellcolor{yellow!20}{\num{-0.009}***} & \cellcolor{yellow!20}{\num{-0.009}***} & \cellcolor{yellow!20}{\num{-0.009}***} & \cellcolor{yellow!20}{\num{-0.011}***} & \cellcolor{yellow!20}{\num{-0.011}***} & \cellcolor{yellow!20}{\num{-0.011}***}\\
 & (\num{0.001}) & (\num{0.001}) & (\num{0.001}) & (\num{0.001}) & (\num{0.001}) & (\num{0.001})\\
\cellcolor{yellow!20}{Unemployment Duration (Binned)} & \cellcolor{yellow!20}{\num{-0.011}***} & \cellcolor{yellow!20}{\num{-0.011}***} & \cellcolor{yellow!20}{\num{-0.008}***} & \cellcolor{yellow!20}{\num{-0.011}***} & \cellcolor{yellow!20}{\num{-0.010}***} & \cellcolor{yellow!20}{\num{-0.007}***}\\
 & (\num{0.002}) & (\num{0.002}) & (\num{0.002}) & (\num{0.002}) & (\num{0.002}) & (\num{0.002})\\
Received Unemployment Compensation &  & \num{-0.000} &  &  & \num{-0.000} & \\
 &  & (\num{0.001}) &  &  & (\num{0.001}) & \\
Exhausted Unemployment Compensation &  &  & \num{0.000}*** &  &  & \num{0.000}***\\
 &  &  & (\num{0.000}) &  &  & (\num{0.000})\\
Female &  &  &  & \num{-0.028}** & \num{-0.028}** & \num{-0.028}**\\
 &  &  &  & (\num{0.011}) & (\num{0.011}) & (\num{0.011})\\
Age &  &  &  & \num{-0.002}*** & \num{-0.002}*** & \num{-0.001}***\\
 &  &  &  & (\num{0.000}) & (\num{0.000}) & (\num{0.000})\\
White &  &  &  & \num{-0.034} & \num{-0.034} & \num{-0.032}\\
 &  &  &  & (\num{0.023}) & (\num{0.023}) & (\num{0.023})\\
Black &  &  &  & \num{-0.057}* & \num{-0.057}* & \num{-0.054}*\\
 &  &  &  & (\num{0.026}) & (\num{0.026}) & (\num{0.026})\\
Mixed &  &  &  & \num{0.017} & \num{0.017} & \num{0.019}\\
 &  &  &  & (\num{0.039}) & (\num{0.039}) & (\num{0.039})\\
Married &  &  &  & \num{0.013} & \num{0.013} & \num{0.013}\\
 &  &  &  & (\num{0.010}) & (\num{0.010}) & (\num{0.010})\\
High School &  &  &  & \num{0.034}* & \num{0.034}* & \num{0.038}*\\
 &  &  &  & (\num{0.015}) & (\num{0.015}) & (\num{0.015})\\
Associate's Degree &  &  &  & \num{0.085}*** & \num{0.085}*** & \num{0.088}***\\
 &  &  &  & (\num{0.021}) & (\num{0.021}) & (\num{0.021})\\
Bachelor's Degree &  &  &  & \num{0.163}*** & \num{0.163}*** & \num{0.166}***\\
 &  &  &  & (\num{0.022}) & (\num{0.022}) & (\num{0.022})\\
Postgraduate Degree &  &  &  & \num{0.246}*** & \num{0.246}*** & \num{0.250}***\\
 &  &  &  & (\num{0.045}) & (\num{0.045}) & (\num{0.045})\\
\midrule
Num.Obs. & \num{4870} & \num{4870} & \num{4870} & \num{4870} & \num{4870} & \num{4870}\\
R2 & \num{0.047} & \num{0.047} & \num{0.051} & \num{0.069} & \num{0.069} & \num{0.072}\\
R2 Adj. & \num{0.047} & \num{0.047} & \num{0.050} & \num{0.067} & \num{0.067} & \num{0.070}\\
F & \num{120.632} & \num{80.422} & \num{86.995} & \num{30.090} & \num{27.774} & \num{29.084}\\
RMSE & \num{0.37} & \num{0.37} & \num{0.37} & \num{0.37} & \num{0.37} & \num{0.37}\\
\bottomrule
\multicolumn{7}{l}{\rule{0pt}{1em}+ p $<$ 0.1, * p $<$ 0.05, ** p $<$ 0.01, *** p $<$ 0.001}\\
\end{tabular}}
\end{table}

\FloatBarrier

Next, we provide results for the econometric specifications listed above
with better balanced survey samples. We outline the various procedures
employed for dealing with selection issues and non-uniformity in the
sample.

\section{Regressions with Selection Correction of Non-Random
Sample}\label{regressions-with-selection-correction-of-non-random-sample}

One of the challenges with this data is that the sample grows
significantly smaller for higher reported of unemployment duration (see
scatter plots in section above). Therefore, we re-weight our survey
sample (beyond the census weights already employed) to ensure population
similarity across bins. More precisely, we employ propensity score
matching using a generalised linear model, entropy-balancing, and
Heckman selection correction.

Overall, we find the econometric results reported earlier to be
consistent across these implementations, with the coefficients on
unemployment duration remaining somewhat stable.

\subsection{Entropy Balancing}\label{entropy-balancing}

First, entropy balancing simply reweights observations to ensure
population matching across the key dependent variable.

\begin{center}\includegraphics[width=0.9\linewidth]{behav_params_overview_files/figure-latex/disc_eb1-1} \end{center}

\FloatBarrier

\begin{center}\includegraphics[width=0.9\linewidth]{behav_params_overview_files/figure-latex/disc_eb2-1} \end{center}

\FloatBarrier

\subsubsection{Diagnostic Tests for Entropy-balanced Reweighted
Sample}\label{diagnostic-tests-for-entropy-balanced-reweighted-sample}

\begin{center}\includegraphics[width=0.9\linewidth]{behav_params_overview_files/figure-latex/disc_eb3-1} \end{center}

\FloatBarrier

\subsection{Propensity Score Weighting with GLM
Estimator}\label{propensity-score-weighting-with-glm-estimator}

\begin{center}\includegraphics[width=0.9\linewidth]{behav_params_overview_files/figure-latex/disc_glm1-1} \end{center}

\FloatBarrier

\subsubsection{Diagnostic Tests for Propensity Score Matching (GLM)
Reweighted
Sample}\label{diagnostic-tests-for-propensity-score-matching-glm-reweighted-sample}

\begin{center}\includegraphics[width=0.9\linewidth]{behav_params_overview_files/figure-latex/disc_glm2-1} \end{center}

\FloatBarrier

\subsubsection{Predicted Reservation Wage using GLM Reweighted
Sample}\label{predicted-reservation-wage-using-glm-reweighted-sample}

\begin{center}\includegraphics[width=0.9\linewidth]{behav_params_overview_files/figure-latex/disc_glm3-1} \end{center}

\FloatBarrier

\subsection{Heckman Selection}\label{heckman-selection}

Additionally, we employ a Heckman Selection correction to correct for
likely selection effects in the data. We correct for selection effects
by balancing across the various control variables (gender, age, race,
marital status, and level of eduction).

\begin{center}\includegraphics[width=0.9\linewidth]{behav_params_overview_files/figure-latex/disc_heckman1-1} \end{center}

\FloatBarrier

\subsection{Regression Results with Sample
Reweighting}\label{regression-results-with-sample-reweighting}

Finally, we provide a comparison of the regression coefficients of the
unbalanced, Heckman corrected, entropy balanced, and GLM reweighting
using propensity score matching. Most importantly, the regression
coefficient on unemployment duration is consistent across specifications
indicating that the consequences of non-uniformity and selection effects
in our sample are minimal. We incorporate the

\begin{table}
\centering\centering
\resizebox{\ifdim\width>\linewidth\linewidth\else\width\fi}{!}{
\begin{tabular}[t]{lcccc}
\toprule
  & Unabalanced LM & Heckman Correction & Entropy Balanced Reweight & GLM Reweight\\
\midrule
Intercept & \num{1.180}*** & \num{1.131}*** & \num{1.147}*** & \num{1.143}***\\
 & (\num{0.031}) & (\num{0.041}) & (\num{0.033}) & (\num{0.033})\\
\cellcolor{yellow!20}{Unemployment Duration (Months)} & \cellcolor{yellow!20}{\num{-0.006}***} & \cellcolor{yellow!20}{\num{-0.006}***} & \cellcolor{yellow!20}{\num{-0.006}***} & \cellcolor{yellow!20}{\num{-0.006}***}\\
 & (\num{0.001}) & (\num{0.001}) & (\num{0.001}) & (\num{0.001})\\
Received Unemployment Compensation & \num{0.000} &  &  & \\
 & (\num{0.001}) &  &  & \\
Female & \num{0.003} & \num{0.018} & \num{0.001} & \num{0.001}\\
 & (\num{0.011}) & (\num{0.014}) & (\num{0.011}) & (\num{0.011})\\
Age & \num{-0.003}*** & \num{-0.007}*** & \num{-0.002}*** & \num{-0.002}***\\
 & (\num{0.000}) & (\num{0.002}) & (\num{0.000}) & (\num{0.000})\\
White & \num{-0.035} & \num{-0.162}* & \num{-0.027} & \num{-0.023}\\
 & (\num{0.023}) & (\num{0.074}) & (\num{0.025}) & (\num{0.025})\\
Black & \num{-0.048}+ & \num{-0.125}* & \num{-0.040} & \num{-0.036}\\
 & (\num{0.026}) & (\num{0.050}) & (\num{0.030}) & (\num{0.030})\\
Mixed & \num{0.014} & \num{-0.054} & \num{0.003} & \num{0.007}\\
 & (\num{0.040}) & (\num{0.055}) & (\num{0.044}) & (\num{0.044})\\
Married & \num{0.005} & \num{0.003} & \num{0.005} & \num{0.004}\\
 & (\num{0.011}) & (\num{0.011}) & (\num{0.011}) & (\num{0.011})\\
High School & \num{0.005} & \num{-0.014} & \num{-0.014} & \num{-0.014}\\
 & (\num{0.016}) & (\num{0.019}) & (\num{0.017}) & (\num{0.017})\\
Associate's Degree & \num{0.032} & \num{-0.078} & \num{0.007} & \num{0.006}\\
 & (\num{0.021}) & (\num{0.064}) & (\num{0.022}) & (\num{0.022})\\
Bachelor's Degree & \num{0.079}*** & \num{-0.217} & \num{0.054}* & \num{0.054}*\\
 & (\num{0.021}) & (\num{0.165}) & (\num{0.023}) & (\num{0.023})\\
Postgraduate Degree & \num{0.114}* & \num{-0.479} & \num{0.083}+ & \num{0.086}+\\
 & (\num{0.045}) & (\num{0.330}) & (\num{0.048}) & (\num{0.047})\\
Inverse Mills Ratio &  & \num{0.870}+ &  & \\
 &  & (\num{0.479}) &  & \\
\midrule
Num.Obs. & \num{4870} & \num{4870} & \num{4870} & \num{4870}\\
R2 & \num{0.025} & \num{0.893} & \num{0.014} & \num{0.015}\\
R2 Adj. & \num{0.022} & \num{0.893} & \num{0.012} & \num{0.013}\\
F & \num{10.220} &  & \num{6.487} & \num{6.798}\\
RMSE & \num{0.37} & \num{0.37} & \num{0.37} & \num{0.37}\\
\bottomrule
\multicolumn{5}{l}{\rule{0pt}{1em}+ p $<$ 0.1, * p $<$ 0.05, ** p $<$ 0.01, *** p $<$ 0.001}\\
\end{tabular}}
\end{table}

Next, we predict the value of the accepted wage ratio using each of the
models, incorporating 95\% confidence intervals to allow for
stochasticity to enter the behavioral mechanism itself. Essentially, as
an agent in our model enters an additional period of unemployment, they
will draw their reservation wage ratio from the mean and 95\% confidence
interval at each unemployment duration value represented in the figure
below. We assume a uniform distribution around the regression estimate
when drawing these values.

\begin{center}\includegraphics[width=0.9\linewidth]{behav_params_overview_files/figure-latex/disc_heckman3-1} \end{center}

\FloatBarrier

\section{Additional Considerations}\label{additional-considerations}

\subsection{Job Tenure}\label{job-tenure}

We have information on the tenure spent at the last job which could
impact the result. This could speak to the ``adaptability'' of
individuals. Wage ratio seems to decrease (although not sure if
meaningfully) with tenure at previous job.

\begin{center}\includegraphics[width=0.9\linewidth]{behav_params_overview_files/figure-latex/unnamed-chunk-9-1} \end{center}

\subsection{Representation}\label{representation}

Although the survey does provide sample weights which we use above, it's
still likely that those who are laid off might be systematically more
susceptible to layoffs (lower-wage, low-skill occupation, male, etc).
Below, we provide some descriptive graphs to illustrate what the sample
looks like. First, the sample over-represents below-mean wage earners
and men. The median age of survey respondents is near the mean age of
the US labor force as reported by the Bureau of Labor Statistics in
2024. Individuals with only a HS diploma represent a strong majority in
the sample.

\begin{center}\includegraphics[width=0.9\linewidth]{behav_params_overview_files/figure-latex/disc_rep-1} \end{center}

\subsection{OTJ Search Propensity}\label{otj-search-propensity}

A key improvement in this model is the incorporation of on-the-job
seekers. Eeckhout et al.~2019 demonstrate that the flow of employed
job-seekers into the pool of job-seekers can generate increased
competition in boom periods. Therefore, we derive a mean value for the
propensity of employed job-seekers to engage in search using the
methodology and data presented in Eeckhout et al.~2019. In other words,
we derive the sensitivity of employed job seekers to the business cycle
from the employment-to-employment transitions data as used in Eeckhout
et al.~Due to unreliable component parts of the Eeckhout analysis, we
decided to abandon using their estimated parameters (search intensity
for employed workers), and instead rely on their series of
employment-to-employment (EE) transition rate which resulted in wage
increases. This series is plotted in the top panel of the following
figure. We draw the mean propensity to search from the average EE
transition rate \textasciitilde6\%, represented by the dark blue dashed
line in the top panel.

We provide additional indicators of the cyclicality of this series by
plotting fitted values of the EE transition rate as a function of
national real GDP. These fitted values are derived from a linear
regression in which the EE transition rate is regressed on national real
GDP, optionally incorporated a deterministic linear trend. Furthermore,
we provide the Hodrick-Prescott filtered series in the bottom panel,
both as the raw de-trended series, and as a fitted series to real GDP.

\begin{figure}
\centering
\pandocbounded{\includegraphics[keepaspectratio]{Eeckhout_Replication/EE_transition_rate.png}}
\caption{Employed Search Effort Fit}
\end{figure}

\subsection{Supporting Data for
Validation}\label{supporting-data-for-validation}

\subsubsection{\texorpdfstring{Intensive Search Effort: Mukoyama et
al.~2018 \emph{Job Search and the Business
Cycle}}{Intensive Search Effort: Mukoyama et al.~2018 Job Search and the Business Cycle}}\label{intensive-search-effort-mukoyama-et-al.-2018-job-search-and-the-business-cycle}

We use evidence from
\href{https://www.aeaweb.org/articles?id=10.1257/mac.20160202}{Mukoyama
et al.~2018} regarding the cyclicality of unemployed job search effort
to validate the micro behavior of our agents. This is a foundational
paper within the literature that explores the relationship between
search effort and business cycles. The authors provide new data on the
intensive margin of unemployed search effort (in minutes searched) over
the business cycle by linking date from the American Time Use Survey
(ATUS) and Current Population Survey (CPS). We employ the methods and
data presented in their work in a validation exercise of the emergent
micro behavior of unemployed job-seekers.

More specifically, the authors provide a novel measure of job search
effort exploiting the American Time Use and Current Population Surveys
which can be reduced to just the intensive margin (changes in search
effort by worker). Typically, this is an extremely challenging measure
to approximate due to data availability and survey design. In general,
surveys measure actions taken (i.e., applications sent, interviews
completed) but these indicators can abstract from the fundamental
intensive margin of search effort. In other words, the most common
metrics that measure search effort can result from passive search or
intrinsic advantageous worker characteristics, obscuring any sense of
real search ``effort'' or urgency with which individuals apply their
search strategies. This intensive margin underlies the motivation behind
our dynamic search effort rule, making this data a valuable source of
validation data for the model's output.

Methodologically, the authors construct this time series measure of job
search intensity by linking the American Time Use Survey (ATUS) and
Current Population Survey (CPS). While the ATUS directly measures
minutes spent on job search activities but has limited sample size and
coverage (2003-2014), the CPS reports the number and types of search
methods used over larger samples beginning in 1994. Both surveys ask
similar questions about search methods employed in the previous month.
The authors exploit this overlap by first estimating the relationship
between reported search time and search methods in the ATUS using a
Heckman selection model---estimating both the probability of positive
search time and the number of minutes conditional on searching,
controlling for demographics, occupation, and unemployment duration.
They then apply these estimated coefficients to impute daily search time
for all CPS unemployed respondents based on their reported search
methods, generating a monthly intensive margin series from 1994-2014.
This approach weights each search method by its estimated time intensity
and allows baseline search effort to vary by demographic
characteristics, producing a more nuanced measure than simply counting
the number of methods used.

The figure below represents the intensive search margin time series as
calculated by the authors. This data is drawn from the replication code
provided by the authors. We have translated the code from Stata to R,
and the time series represented in the figure below relies on the
methodology outlined by Mukoyama et al.~We indicate in the caption and
legend where we have incorporated new data by extending the time series
to include additional years or applied an alternative weighting scheme
to the data to account for missing data. The citation for this work can
be found in the bibliography of the main text.

\begin{figure}
\centering
\pandocbounded{\includegraphics[keepaspectratio]{Mukoyama_Replication/intensive_search_margin.png}}
\caption{Mukoyama Validation Series}
\end{figure}

\subsubsection{Learning Rate - Mueller et al.~Job Seekers' Perceptions
and Employment Prospects: Heterogeneity, Duration Dependence and
Bias}\label{learning-rate---mueller-et-al.-job-seekers-perceptions-and-employment-prospects-heterogeneity-duration-dependence-and-bias}

Given, the theoretical job search model presented in the main text
relies additionally on a learning rate. We consider drawing this
learning rate from work by Mueller et al, though this has not been
incorporated in the main text thus far.

In this work, the authors claim to disentangle the effects of duration
dependence and dynamic selection by using job seekers' elicited beliefs
about job-finding. Assuming (and confirming empirically) that
job-seekers have realistic initial beliefs about job-finding they
isolate the heterogeneity in job-seekers from true duration dependence.
Ultimately, they find that dynamic selection selection explains most of
the negative duration dependence (rather than pure, true duration
dependence).

We replicate and extend the analysis using replication code made
available by the autors. The figures and econometric specification
choices are crafted by Mueller et al, and several of the below plots are
present in the main text of their work as well. Plot and regression
table titles have been maintained for easy comparison. We provide
additional evidence regarding the stability of their estimates across a
longer time series that includes six additional years of data from
2019-2024.

We find that their results are remarkably consistent even when including
additional data from 2019-2024. We aim to include this information in
our theoretical model of the job search effort as a learning rate (ie.
individuals learn about their re-employment probability with repeated
failures in the job search).

\begin{table}[!htbp] \centering 
  \caption{Descriptive Statistics (SCE)} 
  \label{} 
\begin{tabular}{@{\extracolsep{5pt}} cccc} 
\\[-1.8ex]\hline 
\hline \\[-1.8ex] 
Variable & Orig. 2013-19 & 2013-24 & 2020-24 \\ 
\hline \\[-1.8ex] 
High-School Degree or Less & $44.5$ & $40.6$ & $36.9$ \\ 
Some College Education & $32.4$ & $34.9$ & $37.6$ \\ 
College Degree or More & $23.1$ & $24.6$ & $25.6$ \\ 
Age 20-34 & $25.4$ & $27.2$ & $30.0$ \\ 
Age 35-49 & $33.5$ & $33.6$ & $35.3$ \\ 
Age 50-65 & $41.1$ & $39.2$ & $34.8$ \\ 
Female & $59.3$ & $61.2$ & $60.8$ \\ 
Black & $19.1$ & $17.9$ & $16.4$ \\ 
Hispanic & $12.5$ & $13.0$ & $12.6$ \\ 
UE transition rate & $18.7$ & $19.1$ & $18.2$ \\ 
UE transition rate: ST & $25.8$ & $26.5$ & $24.3$ \\ 
UE transition rate: LT & $12.7$ & $12.7$ & $12.3$ \\ 
\# respondents & $948$ & $1,367$ & $433$ \\ 
\# respondents w/ at least 2 u obs & $534$ & $780$ & $252$ \\ 
\# observations & $2,597$ & $3,926$ & $1,347$ \\ 
\hline \\[-1.8ex] 
\end{tabular} 
\end{table}

\begin{center}\includegraphics[width=0.9\linewidth]{behav_params_overview_files/figure-latex/mueller-1} \end{center}

\begin{center}\includegraphics[width=0.9\linewidth]{behav_params_overview_files/figure-latex/mueller-2} \end{center}

\begin{table}[!htbp] \centering 
  \caption{Table 2—Regressions of Realized on Elicited 3-Month Job-Finding Probabilities (SCE): Contemporaneous elicitations} 
  \label{} 
\begin{tabular}{@{\extracolsep{5pt}}lccc} 
\\[-1.8ex]\hline 
\hline \\[-1.8ex] 
 & \multicolumn{3}{c}{\textit{Dependent variable:}} \\ 
\cline{2-4} 
\\[-1.8ex] & \multicolumn{3}{c}{T+3 UE Transitions (3-Months)} \\ 
 & Orig. 2013-19 & 2013-24 & 2020-24 \\ 
\\[-1.8ex] & (1) & (2) & (3)\\ 
\hline \\[-1.8ex] 
 find\_job\_3mon & 0.464$^{***}$ & 0.396$^{***}$ & 0.265$^{***}$ \\ 
  & (0.045) & (0.036) & (0.067) \\ 
  & & & \\ 
 1 \textbar  userid & $-$0.104 &  & $-$0.136 \\ 
  & (0.169) &  & (0.267) \\ 
  & & & \\ 
 Constant &  & $-$0.080 &  \\ 
  &  & (0.137) &  \\ 
  & & & \\ 
\hline \\[-1.8ex] 
Observations & 1,201 & 1,911 & 673 \\ 
R$^{2}$ & 0.218 & 0.139 & 0.105 \\ 
Adjusted R$^{2}$ & 0.207 & 0.132 & 0.083 \\ 
Residual Std. Error & 0.467 (df = 1184) & 0.475 (df = 1894) & 0.478 (df = 656) \\ 
\hline 
\hline \\[-1.8ex] 
\textit{Note:}  & \multicolumn{3}{r}{$^{*}$p$<$0.1; $^{**}$p$<$0.05; $^{***}$p$<$0.01} \\ 
\end{tabular} 
\end{table}

\begin{table}[!htbp] \centering 
  \caption{Table 2—Regressions of Realized on Elicited 3-Month Job-Finding Probabilities (SCE): Contemporaneous elicitations} 
  \label{} 
\begin{tabular}{@{\extracolsep{5pt}}lccc} 
\\[-1.8ex]\hline 
\hline \\[-1.8ex] 
 & \multicolumn{3}{c}{\textit{Dependent variable:}} \\ 
\cline{2-4} 
\\[-1.8ex] & \multicolumn{3}{c}{T+3 UE Transitions (3-Months)} \\ 
 & Orig. 2013-19 & 2013-24 & 2020-24 \\ 
\\[-1.8ex] & (1) & (2) & (3)\\ 
\hline \\[-1.8ex] 
 find\_job\_3mon & 0.501$^{***}$ & 0.418$^{***}$ & 0.391$^{***}$ \\ 
  & (0.061) & (0.051) & (0.094) \\ 
  & & & \\ 
 findjob\_3mon\_longterm & $-$0.258$^{***}$ & $-$0.170$^{**}$ & $-$0.360$^{***}$ \\ 
  & (0.088) & (0.071) & (0.133) \\ 
  & & & \\ 
 longterm\_unemployed & $-$0.078 & $-$0.127$^{***}$ & $-$0.043 \\ 
  & (0.051) & (0.041) & (0.075) \\ 
  & & & \\ 
 1 \textbar  userid &  &  &  \\ 
  &  &  &  \\ 
  & & & \\ 
 Constant & $-$0.062 & $-$0.063 & $-$0.402 \\ 
  & (0.175) & (0.139) & (0.266) \\ 
  & & & \\ 
\hline \\[-1.8ex] 
Observations & 1,201 & 1,911 & 673 \\ 
R$^{2}$ & 0.259 & 0.182 & 0.155 \\ 
Adjusted R$^{2}$ & 0.248 & 0.174 & 0.132 \\ 
Residual Std. Error & 0.455 (df = 1182) & 0.464 (df = 1892) & 0.465 (df = 654) \\ 
\hline 
\hline \\[-1.8ex] 
\textit{Note:}  & \multicolumn{3}{r}{$^{*}$p$<$0.1; $^{**}$p$<$0.05; $^{***}$p$<$0.01} \\ 
\end{tabular} 
\end{table}

\begin{table}[!htbp] \centering 
  \caption{Table 2—Regressions of Realized on Elicited 3-Month Job-Finding Probabilities (SCE): Lagged elicitations} 
  \label{} 
\begin{tabular}{@{\extracolsep{5pt}}lccc} 
\\[-1.8ex]\hline 
\hline \\[-1.8ex] 
 & \multicolumn{3}{c}{\textit{Dependent variable:}} \\ 
\cline{2-4} 
\\[-1.8ex] & \multicolumn{3}{c}{T+3 UE Transitions (3-Months)} \\ 
 & Orig. 2013-19 & 2013-24 & 2020-24 \\ 
\\[-1.8ex] & (1) & (2) & (3)\\ 
\hline \\[-1.8ex] 
 tplus3\_percep\_3mon & 0.332$^{***}$ & 0.241$^{***}$ & 0.203$^{**}$ \\ 
  & (0.067) & (0.056) & (0.102) \\ 
  & & & \\ 
 1 \textbar  userid &  &  &  \\ 
  &  &  &  \\ 
  & & & \\ 
 Constant & 0.304 & 0.490$^{**}$ & 0.451 \\ 
  & (0.270) & (0.207) & (0.394) \\ 
  & & & \\ 
\hline \\[-1.8ex] 
Observations & 474 & 798 & 300 \\ 
R$^{2}$ & 0.168 & 0.090 & 0.179 \\ 
Adjusted R$^{2}$ & 0.139 & 0.071 & 0.132 \\ 
Residual Std. Error & 0.398 (df = 457) & 0.436 (df = 781) & 0.447 (df = 283) \\ 
\hline 
\hline \\[-1.8ex] 
\textit{Note:}  & \multicolumn{3}{r}{$^{*}$p$<$0.1; $^{**}$p$<$0.05; $^{***}$p$<$0.01} \\ 
\end{tabular} 
\end{table}

\begin{table}[!htbp] \centering 
  \caption{Table 2—Regressions of Realized on Elicited 3-Month Job-Finding Probabilities (SCE): Lagged elicitations} 
  \label{} 
\begin{tabular}{@{\extracolsep{5pt}}lccc} 
\\[-1.8ex]\hline 
\hline \\[-1.8ex] 
 & \multicolumn{3}{c}{\textit{Dependent variable:}} \\ 
\cline{2-4} 
\\[-1.8ex] & \multicolumn{3}{c}{T+3 UE Transitions (3-Months)} \\ 
 & Orig. 2013-19 & 2013-24 & 2020-24 \\ 
\\[-1.8ex] & (1) & (2) & (3)\\ 
\hline \\[-1.8ex] 
 find\_job\_3mon & 0.301$^{***}$ & 0.205$^{***}$ & $-$0.035 \\ 
  & (0.069) & (0.058) & (0.110) \\ 
  & & & \\ 
 1 \textbar  userid &  &  &  \\ 
  &  &  &  \\ 
  & & & \\ 
 Constant & 0.201 & 0.422$^{**}$ & 0.361 \\ 
  & (0.274) & (0.207) & (0.400) \\ 
  & & & \\ 
\hline \\[-1.8ex] 
Observations & 474 & 798 & 300 \\ 
R$^{2}$ & 0.159 & 0.083 & 0.168 \\ 
Adjusted R$^{2}$ & 0.129 & 0.064 & 0.121 \\ 
Residual Std. Error & 0.400 (df = 457) & 0.437 (df = 781) & 0.450 (df = 283) \\ 
\hline 
\hline \\[-1.8ex] 
\textit{Note:}  & \multicolumn{3}{r}{$^{*}$p$<$0.1; $^{**}$p$<$0.05; $^{***}$p$<$0.01} \\ 
\end{tabular} 
\end{table}

\begin{center}\includegraphics[width=0.9\linewidth]{behav_params_overview_files/figure-latex/mueller-3} \end{center}

\begin{center}\includegraphics[width=0.9\linewidth]{behav_params_overview_files/figure-latex/mueller-4} \end{center}

\begin{center}\includegraphics[width=0.9\linewidth]{behav_params_overview_files/figure-latex/mueller-5} \end{center}

\begin{center}\includegraphics[width=0.9\linewidth]{behav_params_overview_files/figure-latex/mueller-6} \end{center}

\begin{center}\includegraphics[width=0.9\linewidth]{behav_params_overview_files/figure-latex/mueller-7} \end{center}

\end{document}
